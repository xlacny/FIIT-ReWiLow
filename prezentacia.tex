%TOTO JE PREZENTACIA
% the presentation needs to be run though bibtex, just like the main article !
\documentclass{beamer}

%\usetheme{Warsaw}
%\usetheme{Antibes}
\usetheme{JuanLesPins}
%\usetheme{Goettingen}

%\usecolortheme{seahorse}
%\usecolortheme{dolphin}
%\usecolortheme{rose}
% https://deic.uab.cat/~iblanes/beamer_gallery/
\usecolortheme{beaver}

%\useoutertheme[]{sidebar}

\setbeamercovered{transparent}

%LANGUAGE
\usepackage[T1]{fontenc}
% \usepackage[utf8]{inputenc} % --- ODSTRÁNENÉ (DUPLIKÁT)
\usepackage[slovak]{babel} % pre slovenský jazyk
\usepackage{lmodern} % PRIDANÉ: Pre lepšie fonty s T1
\usepackage{url}
\usepackage{booktabs}
\usepackage{array}
\usepackage{paralist}
\usepackage{verbatim}
\usepackage{subfig}
\usepackage{longtable} % PRIDANÉ: Opravuje chybu 'longtable undefined'
\usepackage{csquotes}
\usepackage{url}
\usepackage{hyperref}
\usepackage[numbers]{natbib}
\usepackage{graphicx} 	

\usepackage{listings}
\usepackage{multirow} % Pre tabuľky

\lstset{language=C++,basicstyle=\fontsize{8}{9.6}\selectfont,showstringspaces=false,columns=fullflexible,identifierstyle=\ttfamily,keywordstyle=\bfseries,showstringspaces=false,columns=fullflexible}
%\lstset{language=C,basicstyle=\fontsize{10.5}{12.6}\selectfont,identifierstyle=\ttfamily,keywordstyle=\bfseries,showstringspaces=false,columns=fixed}

\def\BibTeX{\textsc{Bib}\kern-.08em\TeX} 

\newcommand{\footcite}[1]{\footnote{\tiny #1}}
\newcommand{\umlet}{.5}
\newcommand{\emp}[1]{\textit{\alert{#1}}}
\newcommand{\kw}[1]{\mbox{\textbf{#1}}}
\newcommand{\id}[1]{\texttt{#1}}
\newcommand{\stl}{\guillemotleft}
\newcommand{\str}{\guillemotright}

\newcommand{\lsti}{\lstinline[basicstyle=\fontsize{10.5}{12.1}\selectfont]}

\newcommand{\ssection}[1]{
	\section{#1}
	\begin{frame}[fragile=singleslide]\frametitle{}
	\Huge #1
	\end{frame}
}

\newcommand{\ssectionn}[1]{
	\section*{#1}
	\begin{frame}[fragile=singleslide]\frametitle{}
	\Huge #1
	\end{frame}
}

\newenvironment{program}{\begin{beamercolorbox}[rounded=true,shadow=true]{block body}\vspace{-4mm}}{\vspace{-2mm}\end{beamercolorbox}}

\setbeamercolor{fvystup}{fg=white,bg=black}
\newenvironment{vystup}{\begin{beamercolorbox}[rounded=true,shadow=true]{fvystup}}{\end{beamercolorbox}}

\newenvironment{poznamka}{\begin{beamercolorbox}[rounded=true,shadow=false]{block body}}{\end{beamercolorbox}}

\setbeamertemplate{footline}[page number]
{
%\insertpagenumber
%\begin{beamercolorbox}{section in head/foot}
%\vskip2pt\insertnavigation{\paperwidth}\vskip2pt
%\end{beamercolorbox}%
}



\author{Adam Kšenzulák, Arsenii Leno, Oleh Kysil, Tobias Lačný}
%\url{www.fiit.stuba.sk/~vranic}, \url{vranic@fiit.stuba.sk}}
%{\tiny \url{www.fiit.stuba.sk/~vranic}, \url{vranic@fiit.stuba.sk}}
\institute{
	Ústav informatiky, informačných systémov a softvérového inžinierstva\\
	Fakulta informatiky a informačných technológií\\
	Slovenská technická univerzita v Bratislave}

\subtitle{\vspace{3mm} Metódy inžinierskej práce 2025}

\title{ReWiLow - Prezentácia článku
}

\date{\footnotesize 11. november 2025}




\begin{document}

\begin{frame}[fragile=singleslide]
\titlepage
\end{frame}

% part 1 - ZallasVA - HaLow Introduction (User's original slides)

\begin{frame}{Čo je to HaLow? \cite{wifi_halow2021}}
\begin{itemize}
  \item Štandard \emp{IEEE 802.11ah}
  \item Frekvenčné pásmo 750–950 MHz
  \item \emp{Dlhý dosah}, \emp{nízka spotreba energie}
  \item Až 8191 zariadení na 1 AP
\end{itemize}
\begin{figure}
    \centering
    % Použitie placeholderu pre obrázok
    \framebox{\parbox{0.5\textwidth}{\centering
        \vspace{1cm}
        \textbf{Miesto pre obrázok} \\
        \small\textit{Prehľad štandardu 802.11ah}
        \vspace{1cm}
    }}
    \caption{Dosah HaLow}
    \label{fig:pres_halow01}
\end{figure}
\end{frame}

\begin{frame}{Dizajn HaLow}
\begin{itemize}
  \item Užšie kanály (1–16 MHz)
  \item \emp{Target Wake Time (TWT)} pre optimalizáciu spánku
  \item Zníženie kolízií a šetrná prevádzka
\end{itemize}
\end{frame}

\begin{frame}{Výkon HaLow}
\begin{itemize}
  \item Rýchlosť: 150 kb/s – \emp{78 Mb/s}
  \item Dosah: \emp{do 1 km} (v meste ~ 400 m)
  \item Podpora 8191 zariadení
  \item Energetická úspornosť vďaka nižšej frekvencii
\end{itemize}
\end{frame}

% part 2 - Obsah/Prehľad
\begin{frame}[fragile=singleslide]\frametitle{Prehľad}
\tableofcontents
\end{frame}

% ==========================================================
% ZAČIATOK SEKCIÍ S NOVÝM OBSAHOM OD ARSENA KOZAKA
% ==========================================================

\section{Kľúčové LPWAN Protokoly}

\subsection{LoRaWAN}
\textbf{1. LoRaWAN:} Jeden z najrozšírenejších implementovaných LPWAN protokolov. Využíva moduláciu Chirp Spread Spectrum (CSS) v nelicencovaných frekvenčných pásmach.
\begin{itemize}
    \item Dosah: 2 km (v meste) – 15 km (vo vidieckom prostredí).
    \item Rýchlosť: 0,3 – 27 kbps.
    \item Kľúčová funkcia: Obojsmerná výmena dát (uplink/downlink) a možnosť vytvárať súkromné LoRa siete bez nutnosti operátorov.
\end{itemize}

\subsection{Sigfox}
\textbf{2. SigFox:} Optimalizovaný pre masový IoT (Internet vecí) s veľmi nízkymi dátovými rýchlosťami. Pracuje v rádiových pásmach ISM.
\begin{itemize}
    \item Dosah: až do 40 km.
    \item Veľkosť paketu: 12 bajtov uplink, 8 bajtov downlink.
    \item Nevýhody: Jednosmerný alebo veľmi obmedzený obojsmerný kanál. Vysoká latencia (oneskorenie) – meraná v sekundách.
\end{itemize}

\subsection{NB-IoT (Narrowband IoT)}

\textbf{3. NB-IoT:} Optimalizovaný pre IoT aplikácie s nízkou šírkou pásma, najmä v husto zastavaných mestských prostrediach.
\begin{itemize}
    \item Typické aplikácie: Inteligentné budovy, zámky, pouličné osvetlenie, mestské senzory.
    \item Výhoda: Pracuje na existujúcich mobilných sieťach LTE so zaručenou Kvalitou Služby (QoS - Quality of Service).
\end{itemize}

\subsection{LTE-M / Cat-M1}
\textbf{4. LTE-M:} Funguje na existujúcej infraštruktúre LTE a dosahuje dátové rýchlosti až do 1 Mbps. Umožňuje komunikáciu v reálnom čase s nízkou latenciou. Ideálny pre mobilné zariadenia a nositeľnú elektroniku.
\begin{itemize}
    \item Pozícia: Predstavuje optimálnu rovnováhu medzi rýchlosťou a úsporou energie. Plná podpora mobility.
\end{itemize}

\subsection{Weightless (SIG)}
\textbf{5. Weightless (SIG):} Otvorený štandard LPWAN, navrhnutý pre IoT. Ponúka flexibilné možnosti nasadenia a vďaka svojej otvorenosti umožňuje rôznorodé prípady použitia.
\begin{itemize}
    \item Typ: Otvorený štandard (open standard IoT LPWAN), sub-1 GHz.
    \item Výhoda: Transparentná štruktúra, voľný prístup k štandardu a nekomerčné konzorcium.
\end{itemize}

\subsection{Ingenu (RPMA – Random Phase Multiple Access)}
\textbf{6. Ingenu:} Poskytuje konektivitu na dlhé vzdialenosti s vysokou priepustnosťou. Ponúka robustný sieťový výkon, odolnosť voči rušeniu a pokročilé bezpečnostné funkcie.

\section{Porovnávacia Analýza a Trhové Trendy}

\begin{table}[h!]
    \centering
    \caption{Porovnávacia Matica Parametrov LPWAN Technológií \cite{literatura}}
    \begin{tabular}{l c c c c}
        \toprule
        \textbf{Parameter} & \textbf{NB-IoT} & \textbf{LTE-M} & \textbf{LoRaWAN} & \textbf{Sigfox} \\
        \midrule
        Spektrum & Licencované & Licencované & Nelicencované & Nelicencované \\
        Dosah (mesto/vidiek) & $\approx 1/10$ km & $\approx 1/10$ km & $\approx 5/20$ km & $\approx 10/40$ km \\
        Rýchlosť & 26 -- 66 kbps & až 1 Mbps & 0.3 -- 5.5 kbps & 0.1 kbps \\
        Životnosť batérie & Roky & Roky & Roky & Roky \\
        Latencia & 1.2 -- 10 s & $<$ 60 ms & Sekundy & Sekundy \\
        Mobilita & Obmedzená & Áno & Áno & Áno \\
        Súkromné siete & Nie & Nie & Áno & Nie \\
        Max. Paket & 1280 B & 1280 B & 11 -- 242 B & 12 B (UL) \\
        Výkon & 20 -- 120 mW & 60 -- 200 mW & 25 -- 100 mW & 20 -- 100 mW \\
        \bottomrule
    \end{tabular}
\end{table}

\begin{table}[h!]
    \centering
    \caption{Porovnanie Wi-Fi HaLow s LPWAN \cite{literatura, literatura}}
    \begin{tabular}{l l l}
        \toprule
        \textbf{Kritérium} & \textbf{Wi-Fi HaLow} & \textbf{Porovnanie s LPWAN} \\
        \midrule
        Frekvenčný Rozsah & $<$ 1 GHz (900 MHz) & Lepšia penetrácia cez steny a vegetáciu \\
        Prevádzkový Dosah & do 1 km & Menej ako LoRa/Sigfox, ale viac ako štandardné Wi-Fi \\
        Rýchlosť & až 78 Mbps (krátke vzdial.) & Výrazne vyššia ako LPWAN \\
        Pripojenia & $>$ 8000 zariadení & Škálovateľnosť ako NB-IoT \\
        Latencia & niekoľko ms & Lepšia ako väčšina LPWAN \\
        Spotreba Energie & oveľa menej ako štandardné Wi-Fi & Bližšie k LPWAN \\
        Licencovanie & Nelicencované, otvorený štandard & Nevyžaduje operátorov \\
        \midrule
        \textbf{Pozícia} & \multicolumn{2}{l}{``Most'' medzi Wi-Fi a LPWAN (optimálna rovnováha rýchlosť $\leftrightarrow$ energia)}\\
        \bottomrule
    \end{tabular}
\end{table}

\section{Trhové Trendy a Regionálna Vhodnosť LPWAN}

\begin{table}[h!]
    \centering
    \caption{Trhové Trendy a Regionálna Vhodnosť LPWAN \cite{literatura}}
    \begin{tabular}{l l c l}
        \toprule
        \textbf{Technológia} & \textbf{Hlavné Regióny} & \textbf{Podiel na Trhu} & \textbf{Trendy} \\
        \midrule
        NB-IoT & Čína (84\%), EÚ, Blízky Východ & $\sim 23\%$ (2027) & Rozvoj smart-city, urban-IoT \\
        LTE-M & USA, EÚ, Austrália & $\sim 32\%$ (2023) & Rast vďaka LTE infraštruktúre \\
        LoRaWAN & Sev. Amerika, EÚ, APAC & $\sim 40\%$ (2023), $\rightarrow 35\%$ (2027) & Líder mimo Číny, otvorený ekosystém \\
        Sigfox & EÚ (Francúzsko, Španielsko), Japonsko, Brazília & Menší, ale stabilný & Medzera pre nízkonákladové senzory \\
        \bottomrule
    \end{tabular}
\end{table}


% ==========================================================
% KONIEC NOVÉHO OBSAHU
% ==========================================================


\section*{Zhodnotenie a ďalšia práca}
% hviezdička zabezpečí, aby sa táto časť neocitla v prehľade prezentácie - každá prezentácia má zhodnotenie a prehľad by sa tým zbytočne zahlcoval

\begin{frame}[fragile=singleslide]\frametitle{Zhodnotenie a Ďalšia Práca}
\begin{itemize}
\item Analyzovali sme 6 hlavných LPWAN protokolov a ich pozíciu na trhu.
\item \emp{Wi-Fi HaLow} predstavuje kľúčovú medzeru medzi klasickým Wi-Fi a LPWAN technológiami vďaka optimálnej rovnováhe rýchlosti a dosahu.
\item \kw{Ďalšia práca:} Experimentálne overenie výkonu HaLow v reálnych podmienkach (penetrácia, latencia) a porovnanie so zavedenými štandardmi ako LoRaWAN.
\end{itemize}
\end{frame}
\section*{Zdroje}
\bibliographystyle{unsrt}
\bibliography{literatura}
\end{document}
