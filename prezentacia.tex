%TOTO JE PREZENTACIA
% the presentation needs to be run though bibtex, just like the main article !
\documentclass{beamer}

%\usetheme{Warsaw}
%\usetheme{Antibes}
\usetheme{JuanLesPins}
%\usetheme{Goettingen}

%\usecolortheme{seahorse}
%\usecolortheme{dolphin}
%\usecolortheme{rose}
% https://deic.uab.cat/~iblanes/beamer_gallery/
\usecolortheme{beaver}

%\useoutertheme[]{sidebar}

\setbeamercovered{transparent}

%LANGUAGE
\usepackage[T1]{fontenc}
% \usepackage[utf8]{inputenc} % --- ODSTRÁNENÉ (DUPLIKÁT)
\usepackage[slovak]{babel} % pre slovenský jazyk
\usepackage{lmodern} % PRIDANÉ: Pre lepšie fonty s T1
\usepackage{url}
\usepackage{booktabs}
\usepackage{array}
\usepackage{paralist}
\usepackage{verbatim}
\usepackage{subfig}
\usepackage{longtable} % PRIDANÉ: Opravuje chybu 'longtable undefined'
\usepackage{csquotes}
\usepackage{url}
\usepackage{hyperref}
\usepackage[numbers]{natbib}
\usepackage{graphicx} 	

\usepackage{listings}
\usepackage{multirow} % Pre tabuľky

\lstset{language=C++,basicstyle=\fontsize{8}{9.6}\selectfont,showstringspaces=false,columns=fullflexible,identifierstyle=\ttfamily,keywordstyle=\bfseries,showstringspaces=false,columns=fullflexible}
%\lstset{language=C,basicstyle=\fontsize{10.5}{12.6}\selectfont,identifierstyle=\ttfamily,keywordstyle=\bfseries,showstringspaces=false,columns=fixed}

\def\BibTeX{\textsc{Bib}\kern-.08em\TeX} 

\newcommand{\footcite}[1]{\footnote{\tiny #1}}
\newcommand{\umlet}{.5}
\newcommand{\emp}[1]{\textit{\alert{#1}}}
\newcommand{\kw}[1]{\mbox{\textbf{#1}}}
\newcommand{\id}[1]{\texttt{#1}}
\newcommand{\stl}{\guillemotleft}
\newcommand{\str}{\guillemotright}

\newcommand{\lsti}{\lstinline[basicstyle=\fontsize{10.5}{12.1}\selectfont]}

\newcommand{\ssection}[1]{
	\section{#1}
	\begin{frame}[fragile=singleslide]\frametitle{}
	\Huge #1
	\end{frame}
}

\newcommand{\ssectionn}[1]{
	\section*{#1}
	\begin{frame}[fragile=singleslide]\frametitle{}
	\Huge #1
	\end{frame}
}

\newenvironment{program}{\begin{beamercolorbox}[rounded=true,shadow=true]{block body}\vspace{-4mm}}{\vspace{-2mm}\end{beamercolorbox}}

\setbeamercolor{fvystup}{fg=white,bg=black}
\newenvironment{vystup}{\begin{beamercolorbox}[rounded=true,shadow=true]{fvystup}}{\end{beamercolorbox}}

\newenvironment{poznamka}{\begin{beamercolorbox}[rounded=true,shadow=false]{block body}}{\end{beamercolorbox}}

\setbeamertemplate{footline}[page number]
{
%\insertpagenumber
%\begin{beamercolorbox}{section in head/foot}
%\vskip2pt\insertnavigation{\paperwidth}\vskip2pt
%\end{beamercolorbox}%
}



\author{Adam Kšenzulák, Arsenii Leno, Oleh Kysil, Tobias Lačný}
%\url{www.fiit.stuba.sk/~vranic}, \url{vranic@fiit.stuba.sk}}
%{\tiny \url{www.fiit.stuba.sk/~vranic}, \url{vranic@fiit.stuba.sk}}
\institute{
	Ústav informatiky, informačných systémov a softvérového inžinierstva\\
	Fakulta informatiky a informačných technológií\\
	Slovenská technická univerzita v Bratislave}

\subtitle{\vspace{3mm} Metódy inžinierskej práce 2025}

\title{ReWiLow - Prezentácia článku
}

\date{\footnotesize 11. november 2025}


\begin{document}

\begin{frame}[fragile=singleslide]
\titlepage
\end{frame}

\begin{frame}[fragile=singleslide]\frametitle{Prehľad}
\tableofcontents
\end{frame}

% part 1 - ZallasVA - HaLow Introduction (User's original slides)

\section {Základné informácie o HaLow}

\begin{frame}{Čo je to HaLow? \cite{wifi_halow2021}}
\begin{itemize}
  \item Štandard \emp{IEEE 802.11ah}
  \item Frekvenčné pásmo 750-950 MHz
  \item \emp{Dlhý dosah}, \emp{nízka spotreba energie}
\end{itemize}
\begin{figure}[r]
    \centering
    \includegraphics[width=0.5\textwidth]{80211ah_overview.png} 
    % \vspace{3cm}
        % \textbf{Miesto pre obrázok} \\
        % \small\textit{LPWAN sieť (obrázok LPWAN\_pc1.png)}
        % \vspace{3cm}
    \caption{Dosah HaLow}
    \label{fig:pres_halow01}
\end{figure}
\end{frame}

\begin{frame}{Dizajn HaLow}
\begin{itemize}
  \item Užšie kanály (1-16 MHz)
  \item \emp{Target Wake Time (TWT)} pre optimalizáciu spánku
  \item Zníženie kolízií a šetrná prevádzka
\end{itemize}
\begin{figure}[r]
    \centering
    \includegraphics[width=0.5\textwidth]{target-wake-time-wifi-6-6e} 
    % \vspace{3cm}
        % \textbf{Miesto pre obrázok} \\
        % \small\textit{LPWAN sieť (obrázok LPWAN\_pc1.png)}
        % \vspace{3cm}
    \caption{Target Wake Time \cite{twt}}
    \label{fig:pres_halow02}
\end{figure}
\end{frame}

\begin{frame}{Výkon HaLow}
\begin{itemize}
  \item Rýchlosť: 150 kb/s - \emp{78 Mb/s}
  \item Dosah: \emp{do 1 km} (v meste ~ 400 m)
  \item Energetická úspornosť vďaka nižšej frekvencii
  \item Až 8191 zariadení na 1 AP
\end{itemize}
\end{frame}

% part 2 - Nereg - uniqueness of the HaLow standard

% slide 1 - L1 - free frequencies, greater penetration and reach, cheaper equipment

\begin{frame}{Výnimočnosť 802.11ah - L1}
  \hfil\hfil\includegraphics[height=0.65\paperheight]{Oleh_prez_range_performance_comp.jpg}\newline
  \null\hfil\hfil\makebox[0.5\linewidth]{"Širši"  výkon}\newline
\end{frame}

\begin{frame}{Výnimočnosť 802.11ah - L1}
  \hfil\hfil\makebox[5cm]{Lacnejšie zariadenia}\newline
  \hfil\hfil\includegraphics[width=0.5\linewidth]{Oleh_prez_802.11ah_device.png}\hfil\hfil
  \includegraphics[width=0.5\linewidth]{Oleh_prez_LTE_device.png}\newline
  \hfil\makebox[5cm]{802.11ah: 60 €}
  \hfil\makebox[5cm]{NB-IoT: "Request pricing"}
%   \hfil\hfil\makebox[5cm]{Cheaper equipment}
\end{frame}

% slide 2 - higher levels - built-in tried and true technologies (WPA3, TCP/IP), device class mixing

\begin{frame}{Výnimočnosť 802.11ah - výšie vrstvy}
  \hfil\hfil\makebox[5cm]{Overené a spoľahlivé technológie integrované do štandardu}\newline
  \hfil\hfil\includegraphics[width=0.5\linewidth]{Oleh_prez_device_interop.png}\hfil\hfil
  \includegraphics[width=0.5\linewidth]{Oleh_prez_TCP_IP.png}\newline
  \hfil\makebox[5cm]{Komunikácia rôznych typov zariadení}
  \hfil\makebox[5cm]{Integrovana vrstva TCP/IP}
\end{frame}


% ==========================================================
% ZAČIATOK SEKCIÍ S NOVÝM OBSAHOM OD ARSENA KOZAKA
% ==========================================================

% --- ČASŤ 2: LPWAN ---
\section{Kľúčové LPWAN Protokoly}
\begin{frame}{Kľúčové LPWAN Protokoly \cite{5gtechworld2024_lpwan}}
  \begin{itemize}
    \item \textbf{LoRaWAN:} Jeden z najrozšírenejších LPWAN protokolov.
    \item \textbf{SigFox:} Optimalizovaný pre masový IoT.
    \item \textbf{NB-IoT:} Pre husté mestské prostredia.
    \item \textbf{LTE-M / Cat-M1:} Ideálny pre mobilné zariadenia.
    \item \textbf{Weightless (SIG):} Pre rôznorodé prípady použitia.
    \item \textbf{Ingenu:} Dlhý dosah s vysokou priepustnosťou.
  \end{itemize}
\end{frame}


% --- ČASŤ 3: Porovnania ---
\section{Porovnávacia Analýza LPWAN}
\begin{frame}{Porovnávacia Matica Parametrov LPWAN \cite{dfrobot2025_lpwan}}
  \begin{table}[h!]
    \centering
    \setlength{\tabcolsep}{3pt} % zmenší horizontálne medzery
    \renewcommand{\arraystretch}{0.9} % zmenší výšku riadkov
    \scriptsize % menší font, aby sa zmestilo na slide
    \begin{tabular}{p{2.3cm} c c c c}
      \toprule
      \textbf{Parameter} & \textbf{NB-IoT} & \textbf{LTE-M} & \textbf{LoRaWAN} & \textbf{Sigfox} \\
      \midrule
      Spektrum & Licencované & Licencované & Nelicencované & Nelicencované \\
      Dosah (mesto/vidiek) & 1/10 km & 1/10 km & 5/20 km & 10/40 km \\
      Rýchlosť & 26–66 kbps & až 1 Mbps & 0.3–5.5 kbps & 0.1 kbps \\
      Životnosť batérie & Roky & Roky & Roky & Roky \\
      Latencia & 1.2–10 s & <60 ms & Sekundy & Sekundy \\
      Mobilita & Obmedzená & Áno & Áno & Áno \\
      Súkromné siete & Nie & Nie & Áno & Nie \\
      Max. paket & 1280 B & 1280 B & 11–242 B & 12 B (UL) \\
      Výkon & 20–120 mW & 60–200 mW & 25–100 mW & 20–100 mW \\
      \bottomrule
    \end{tabular}
  \end{table}
\end{frame}


\section{Perspektíva Wi-Fi HaLow}
\begin{frame}{Porovnanie Wi-Fi HaLow s LPWAN \cite{silex_technology_lpwan}}
  \begin{table}
    \centering
    \begin{tabular}{l l l}
      \toprule
      \textbf{Kritérium} & \textbf{Wi-Fi HaLow} & \textbf{Porovnanie s LPWAN} \\
      \midrule
      Frekvenčný Rozsah & < 1 GHz & Lepšia penetrácia \\
      Prevádzkový Dosah & do 1 km & Menej ako LoRa/Sigfox \\
      Rýchlosť & až 78 Mbps & Vyššia ako LPWAN \\
      Pripojenia & > 8000 zariadení & Škálovateľnosť ako NB-IoT \\
      Latencia & niekoľko ms & Lepšia než väčšina LPWAN \\
      Spotreba Energie & menej ako Wi-Fi & Bližšie k LPWAN \\
      Licencovanie & Nelicencované & – \\
      \midrule
      \textbf{Pozícia} & \multicolumn{2}{l}{„Most” medzi Wi-Fi a LPWAN (speed $\leftrightarrow$ energy)} \\
      \bottomrule
    \end{tabular}
  \end{table}
\end{frame}

\begin{frame}{Wi-Fi HaLow: Univerzálny Nástroj \cite{newracom_disrupt}}
  \Large
  \begin{itemize}
    \item \textbf{Wi-Fi HaLow} – \textit{univerzálny nástroj}, ktorý funguje ako \textit{\color{red}hybrid}.
    \vspace{0.8cm}
    \item Vzal si to najlepšie z oboch svetov:
      \begin{itemize}
        \large
        \item \textbf{Od LPWAN:} Lepší \textit{\color{red}dosah} a penetráciu (vďaka sub-GHz).
        \item \textbf{Od klasického Wi-Fi:} Vyššiu \textit{rýchlosť} a nízku \textit{latenciu}.
      \end{itemize}
    \item Optimálny \textbf{kompromis} medzi energiou a rýchlosťou.
  \end{itemize}
\end{frame}

% --- ČASŤ 4: Trhové trendy ---
\section{Trhové Trendy a Regionálna Vhodnosť LPWAN}
\begin{frame}{Trhové Trendy a Regionálna Vhodnosť LPWAN \cite{iotanalytics2018_lpwan}}
  \begin{table}[h!]
    \centering
    \setlength{\tabcolsep}{0.3pt}
    \begin{tabular}{l l c l}
      \toprule
      \textbf{Technológia} & \textbf{Hlavné Regióny} & \textbf{Podiel na trhu} & \textbf{Trendy} \\
      \midrule
      NB-IoT & Čína (84\%), EÚ, Blízky Východ & ~23\% (2027) & Smart-city, urban-IoT \\
      LTE-M & USA, EÚ, Austrália & ~32\% (2023) & Rast vďaka LTE infraštruktúre \\
      LoRaWAN & Sev. Amerika, EÚ, APAC & ~40\% (2023) → 35\% (2027) & Líder mimo Číny \\
      Sigfox & EÚ, Japonsko, Brazília & Menší, ale stabilný & Pre lacné senzory \\
      \bottomrule
    \end{tabular}
  \end{table}
\end{frame}


% ==========================================================
% KONIEC NOVÉHO OBSAHU
% ==========================================================

\begin{frame}{Dominantné Aplikačné Scenáre}
    \frametitle{Kde by si HaLow mohol v budúcnosti využívať}

    \begin{itemize}
        \item \textbf{Priemyselný IoT (IIoT):}
        \begin{itemize}
            \item Spoľahlivé pokrytie pre \textbf{sledovanie majetku} (Asset Tracking) a \textbf{video dohľad} v halách.
        \end{itemize}
        \item \textbf{Smart Budovy/Kampusy:}
        \begin{itemize}
            \item Komplexné riadenie (HVAC, energetický manažment) a rozsiahle senzorové siete.
        \end{itemize}
        \item \textbf{Inteligentné Poľnohospodárstvo:}
        \begin{itemize}
            \item Pokrytie veľkých plôch s rýchlosťou pre \textbf{monitoring a automatizáciu}.
        \end{itemize}
    \end{itemize}


\end{frame}


\section*{Zhodnotenie a ďalšia práca}
% hviezdička zabezpečí, aby sa táto časť neocitla v prehľade prezentácie - každá prezentácia má zhodnotenie a prehľad by sa tým zbytočne zahlcoval

\begin{frame}[fragile=singleslide]\frametitle{Zhodnotenie a Ďalšia Práca}
\begin{itemize}
\item Analyzovali sme 6 hlavných LPWAN protokolov a ich pozíciu na trhu.
\item \emp{Wi-Fi HaLow} predstavuje kľúčovú medzeru medzi klasickým Wi-Fi a LPWAN technológiami vďaka optimálnej rovnováhe rýchlosti a dosahu.
\item \kw{Ďalšia práca:} Experimentálne overenie výkonu HaLow v reálnych podmienkach (penetrácia, latencia) a porovnanie so zavedenými štandardmi ako LoRaWAN.
\end{itemize}
\end{frame}
\section*{Zdroje}
\bibliographystyle{unsrt}
\bibliography{literatura}
\end{document}
