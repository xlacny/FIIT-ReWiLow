% !TEX TS-program = pdflatex
% !TEX encoding = UTF-8 Unicode

\documentclass[11pt]{article}

\usepackage[utf8]{inputenc}
\usepackage{amsmath}

%LANGUAGE
\usepackage[T1]{fontenc}
% \usepackage[utf8]{inputenc} % --- ODSTRÁNENÉ (DUPLIKÁT)
\usepackage[slovak]{babel} % pre slovenský jazyk
\usepackage{lmodern}         % --- PRIDANÉ: Pre lepšie fonty s T1

%DIMENZIE

\usepackage{geometry}
\geometry{
    a4paper,
    top=2.5cm,      % Pole hore
    bottom=2cm,     % Pole dole
    left=3cm,       % Ľavé pole
    right=1.5cm,    % Prave pole
    headheight=14pt % Výška colontitula 
}

\usepackage{graphicx} 
% \usepackage[IL2]{fontenc} % --- ODSTRÁNENÉ (Konflikt s T1)

%PACKAGES
\usepackage{booktabs} 
\usepackage{array} 
\usepackage{paralist} 
\usepackage{verbatim} 
\usepackage{subfig} 
\usepackage{longtable} % --- PRIDANÉ: Opravuje chybu 'longtable undefined'
\usepackage{csquotes}
\usepackage[hyphens]{url}
\usepackage{hyperref}
\usepackage[numbers]{natbib}

% HEADERS & FOOTERS
\usepackage{fancyhdr}
\pagestyle{fancy} 
\renewcommand{\headrulewidth}{0pt}
\lhead{}\chead{}\rhead{}
\lfoot{}\cfoot{\thepage}\rfoot{}

\usepackage{sectsty}
\allsectionsfont{\sffamily\mdseries\upshape}

\usepackage[nottoc,notlof,notlot]{tocbibind} 
\usepackage[titles,subfigure]{tocloft} 
\renewcommand{\cftsecfont}{\rmfamily\mdseries\upshape}
\renewcommand{\cftsecpagefont}{\rmfamily\mdseries\upshape}

\title{ \textbf{Výskum WiFi HaLow Technológie}}
\author{Adam Kšenzulák\and Arsenii Leno\and Oleh Kysil\and Tobias Lačný}
\date{}

\begin{document}
\maketitle
\paragraph {\textbf{Akronym:} \texttt{ReWiLow}} % Upravené \bf a \tt
\begin{center}
{\section*{\textbf{Abstrakt}} }% Upravené \bf
\end{center}
Naša skupina skúma štandard IEEE 802.11ah pre LPWAN (Low Power Wide Area Network – Nízkoenergetická rozsiahla sieť), bežne známy ako Wi-fi HaLow, so zameraním  na jeho architektonický dizajn, výkonové charakteristiky a vhodnosť pre komunikáciu v rámci Internetu vecí (IoT), komuníkaciu medzi strojmi (M2M) aj na trhu. Skúmané sú aj jedinečné technické vlastnosti HaLow, ako pre prevádzka pod 1GHz v nelicencovaných rádiových (ISM) pásiem, rozsšírený dosah, nízka spotreba energie, a aj vysoká hustota zariadení. Hodnotená je aj jeho použiteľnosť v reálnych scenároch, vrátane inteligentného poľnohospodárstva, priemyselnej automatizácie, monitorovania životného prostredia a inteligentných miest.
V rámci projektu sa bude zameriavať aj na komparatívnu analýzu medzi Wi-Fi HaLow a inými LPWAN technológiami, príkladom sú LoRaWAN, Sigfox, alebo aj NB IoT najmä v ich parametroch, ako sú dosah, rýchlosť prenosu dát, latencia, energetická účinnosť a škálovateľnosť siete. Našim cieľom bude určiť konkrétne kontexty, v ktorých Wi-Fi HaLow ponúka výhody alebo obmedzenia v porovnaní s konkurečnými riešeniami LPWAN. Výsledky majú slúžiť ako podklad pre budúci rozvoj infraštruktúry IoT a stratégií konektivity na Slovensku
\\ \\ \\ \\ \\

\begin{center}
\textit{Projekt je pod vedením pána\\ Ing. Ivana Kapustíka. Akademický rok 2025/2026 - Metódy inžinierskej práce, Skupina 6}
\end{center}
\newpage

\section*{\textbf{Všeobecné údaje}} % Upravené \bf

\subsection*{\textbf{Nízkoprúdové Rozsiahle Siete (LPWAN)}} % Upravené \bf

\noindent
\textbf{Nízkoprúdová rozsiahla sieť} (Low-Power, Wide-Area Network – \textbf{LPWAN} alebo \textbf{LPWA}) predstavuje špecifický typ bezdrôtovej telekomunikačnej rozsiahlej siete. Jej primárnym účelom je umožniť \textbf{komunikáciu na dlhé vzdialenosti} pri \textbf{nízkej bitovej rýchlosti}. Táto architektúra je navrhnutá predovšetkým na pripojenie zariadení \textbf{internetu vecí (IoT)}, ako sú senzory, ktoré sú typicky napájané z batérie.

\vspace{0.5\baselineskip} % Malý vertikálny priestor medzi odsekmi

\noindent
Charakteristické vlastnosti, ako sú \textbf{nízka spotreba energie}, obmedzená prenosová rýchlosť a špecifické určenie použitia, odlišujú siete LPWAN od klasických bezdrôtových rozsiahlych sietí (\textbf{WWAN}). Klasické WWAN sú totiž optimalizované na pripojenie koncových používateľov alebo podnikov a sú dimenzované na prenos väčšieho objemu dát pri vyššej spotrebe energie. Dátová priepustnosť v sieťach LPWAN sa obvykle pohybuje v rozmedzí \textbf{od $0.3\ \text{kbit/s}$ do $50\ \text{kbit/s}$} na kanál.

\vspace{0.5\baselineskip}

\noindent
LPWAN technológie umožňujú vytvorenie \textbf{privátnych bezdrôtových senzorových sietí}. Zároveň môžu byť ponúkané ako \textbf{infraštruktúrna služba} tretej strany. Táto možnosť eliminuje pre majiteľov senzorov nutnosť investovať do vlastnej \textbf{gateway} (\textbf{bránovej}) technológie, čím sa zjednodušuje a urýchľuje nasadenie zariadení v teréne.

\subsection*{\textbf{Princíp fungovania LPWAN}} % Upravené \bf

\noindent
Základom princípu prenosu dát technológiou LPWAN na fyzickej vrstve PHY je \textbf{vlastnosť rádiových systémov} – \textit{zvýšenie energie,} a tým aj dosahu komunikácie pri \textit{znížení prenosovej rýchlosti}. Čím \textbf{nižšia je prenosová bitová rýchlosť}, tým viac energie sa vkladá do každého bitu, a tým \textbf{ľahšie je ho rozlíšiť} na pozadí šumu v prijímacej časti systému. Nízka prenosová rýchlosť dát teda umožňuje dosiahnuť väčší dosah ich príjmu.

\vspace{0.5\baselineskip}

\noindent
Prístup používaný na konštrukciu LPWAN-siete je \textbf{podobný} princípu fungovania \textbf{mobilných komunikačných sietí}. LPWAN-sieť využíva topológiu \textit{„hviezda“}, kde každé zariadenie komunikuje priamo so základňovou stanicou. Mestské alebo regionálne siete sa konštruujú s použitím konfigurácie „hviezda z hviezd“.

% --- OBRÁZOK NAHRADENÝ ZAPISOVAČOM ---
\begin{figure}[h!]
    \centering
    \includegraphics[width=0.5\textwidth]{LPWAN_pc1.png} 
    % \vspace{3cm}
        % \textbf{Miesto pre obrázok} \\
        % \small\textit{LPWAN sieť (obrázok LPWAN\_pc1.png)}
        % \vspace{3cm}
    \caption{LPWAN sieť}
    \label{fig:pc1}
\end{figure}
% --- KONIEC ZAPISOVAČA ---

\newpage
%Tobias
\section{\textbf{Wi-fi HaLow - Konkrétne informácie}}
\textit{Autor: Tobias Lačný}

V informatickom modernom svete sa čoraz viac a viac začínajú využívať produkty IoT (Internetu vecí), Množstvo produktov, od bezpečnostných kamier, smart chladiace prístrojov, smart mikrovlnné trúb až po jednoduchú komunikácia na rádiových zariadeniach  \cite{wifi_halow2021}.

Wi-fi HaLow je štandard nadizajnovaný, oproti iným Wi-fi štandardom, na ďaleký prenos, vysokú energetickú úsporu a oproti rádiovým vlnám aj vyšší dátový prenos pre malý počet zariadení \cite{wifi_halow2021}. Technológie, ktoré majú implementovanú technológiu IEEE 802.11ah sa označujú ako Wi-fi HaLow (tm).

Väčší dosah štandardu Wi-fi HaLow  je vďaka fungovaniu v sub-1 GHz frekvenčnom pásme, zvyčajne 750 až 950 MHz. Spotreba energie je priamo úmerná frekvencií, teda Wi-fi HaLow štandard ušetrí viac energie \cite{quectel2023_halow}.

\subsection{\textbf{Výkon Wi-Fi HaLow}}
IEEE 802.11ah má oproti minulým štandardom rovnakého názvu možnosť vyhradiť určité časové úseky, pri ktorých môže zariadenie, ľudovo povedané, " spať" , nespôsobujúc problémy s kolíziou signálov či lepšie rozdelenie vysielaní vysielačov \cite{mdpi_halow_01} 
Takáto funkcia sa nazýva \textbf{Tager Wake Time} (TWT), zariadenia sa teda prebúdzajú len v určených intervaloch. \cite{ieee_access_2019_raw}

Vďaka podpore frekvenčne užších kanálov (1 až 16 MHz) je možné dosahovať pomerne výhodné rýchlosti od 150 kb/s až do 78 Mb/s, závisiac od konfigurácie. \cite{wifi_halow2021}
Dosah signálu sa pohybuje v intervalu až do 1 km, záležiac od prekážok. Napríklad v meste by signal dosahoval približne 400 metrov.
Štandard naviac podporuje až \textbf{8191} pripojených zariadení na jeden prístupný bod, čo prevyšuje limity klasických Wi-Fi technológií \cite{mdpi_halow_01} 


% -------- porovnanie so súčasnými technológiami @Άρςεν_Κόζακ ⚜ (štandardizované ako 2.4) ---

\section{Porovnanie LPWAN s inými modernými technológiami}
\label{sec:lpwan_vs_others}

\subsubsection*{Filozofia LPWAN}
\begin{itemize}
    \item Nie rýchlosť, ale úspora energie + pokrytie + životnosť.
\end{itemize}

\noindent
Väčšina LPWAN protokolov nedokáže podporovať aplikácie s vysokou šírkou pásma, ako je streamovanie videa alebo rozsiahle, rýchle prenosy dát. \textbf{LTE-M} napríklad ponúka vyššie rýchlosti dát, podporujúc \textit{Voice over LTE (VoLTE)} pre podnikové a priemyselné nasadenia. \cite{iotanalytics2018_lpwan}

\noindent
Hoci LPWAN rádiá umožňujú spoľahlivú obojsmernú komunikáciu, niektoré implementácie sú náchylné na latenciu a oneskorenia prenosu, čo obmedzuje ich efektivitu pre riadiace aplikácie v reálnom čase, ktoré vyžadujú rýchle, veľké prenosy dát a nízku alebo ultra-nízku latenciu. Avšak, LPWAN exceluje v rozsiahlych nasadeniach s nízkou šírkou pásma, ktoré vyžadujú komunikáciu na veľké vzdialenosti a minimálnu spotrebu energie (napr. siete distribuovaných zariadení, ktoré prerušovane posielajú malé dátové pakety).

\noindent
Naopak, \textbf{Wi-Fi}, \textbf{Bluetooth LE} a \textbf{ZigBee} sa neškálujú efektívne pre nepretržitú, rozsiahlu konektivitu M2M (Machine-to-Machine) a IIoT (Industrial IoT). Tieto protokoly sú však široko používané v mnohých spotrebiteľských, podnikových a priemyselných aplikáciách. Napríklad, \textbf{Wi-Fi} je preferovanou voľbou pre kamery s vysokou priepustnosťou a video zvončeky, pričom štandardy Wi-Fi 6, 6E a 7 implementujú funkcie ako \textit{Target Wake Time (TWT)} a \textit{Power Save Mode (PSM)} na optimalizáciu spotreby energie.

\noindent
\textbf{Bluetooth LE} je populárny pre fitness trackery, medicínske zariadenia a pripojené systémy osvetlenia. Zatiaľ čo problémy s interferenciou a fragmentované profily spomalili adopciu \textbf{ZigBee}, tento protokol je stále životaschopný pre domácu automatizáciu, inteligentné osvetlenie a environmentálne senzory.
\cite{dfrobot2025_lpwan}

\subsubsection{Detailný Prehľad Kľúčových LPWAN Protokolov \cite{dfrobot2025_lpwan}}
\label{sec:detailed_lpwan_review}
\textit{Autor: Arsen Kozak}

\vspace{0.5\baselineskip}
\subsubsection*{1. LoRaWAN}
\noindent
\textbf{LoRaWAN} je jeden z najrozšírenejších implementovaných LPWAN protokolov. Pracuje na modulácii \textit{chirp spread spectrum (CSS)} v nelicencovaných frekvenčných pásmach, kde LoRaWAN zariadenia komunikujú priamo s bránami, ktoré posielajú dáta na centrálne servery. Zabezpečený šifrovaním \textbf{AES-128}, tento populárny protokol poskytuje dátové rýchlosti od $0.3$ kbps do $27$ kbps. Prenosové dosahy sa typicky pohybujú do 15 km vo vidieckych oblastiach a približne $1-2$ km v mestskom prostredí. V závislosti od scenárov použitia môžu zariadenia LoRaWAN fungovať roky na jednu batériu.

\begin{itemize}[$\bullet$]
    \item Typ: Nelicencovaný, založený na CSS modulácii (\textit{Chirp Spread Spectrum})
    \item Dosah: 2 km (mesto) – 15 km (vidiek); Rýchlosť: $0.3-27$ kbps
    \item \textbf{Funkcia:} Obojsmerná výmena dát (\textit{uplink/downlink}) a možnosť súkromných LoRa sietí bez operátorov.
\end{itemize}

\vspace{0.5\baselineskip}
\subsubsection*{2. Sigfox}
\noindent
\textbf{Sigfox} je úzkopásmová, necelulárna technológia optimalizovaná pre masový IoT s nízkymi dátovými rýchlosťami. Pracujúc v rádiových pásmach ISM, Sigfox ponúka široký prenosový dosah a nízke prevádzkové náklady. Zariadenia vysielajú správy viacerým základňovým staniciam (v priemere tri stanice prijmú každý prenos). Zameraný na aplikácie s nízkou šírkou pásma, Sigfox zaisťuje dlhú životnosť batérie a minimálne náklady na zariadenie vďaka jednoduchým hardvérovým komponentom a nízkoenergetickým polovodičom.

\begin{itemize}[$\bullet$]
    \item Typ: Nelicencované ISM, Ultra-úzke pásmo (Ultra-Narrow Band)
    \item Dosah: do 40 km; Paket: 12 bajtov \textit{uplink}, 8 bajtov \textit{downlink}
    \item \textbf{Nevýhody:} Jednosmerný alebo obmedzený obojsmerný kanál, latencia v sekundách.
\end{itemize}
\vspace{0.5\baselineskip}
\subsubsection*{3. NB-IoT (\textit{Narrowband IoT})}
\noindent
\textbf{NB-IoT} je celulárny LPWAN protokol optimalizovaný pre IoT aplikácie s nízkou šírkou pásma v hustých mestských prostrediach. Pracuje so šírkou kanála $180$ kHz, poskytuje dátové rýchlosti do $30$ Kbps (downlink) a $60$ Kbps (uplink), pričom sa vyznačuje optimálnou penetráciou v interiéroch, minimálnou interferenciou a dlhou životnosťou batérie.

\begin{itemize}[$\bullet$]
    \item Typ: Celulárny (Licencovaný); Šírka kanála: 180 kHz
    \item Aplikácia: Inteligentné budovy, zámky, pouličné osvetlenie, mestské senzory.
    \item \textbf{Výhoda:} Pracuje na existujúcich LTE sieťach so zaručenou QoS (Quality of Service).
\end{itemize}

\vspace{0.5\baselineskip}
\subsubsection*{4. LTE-M / Cat-M1}
\noindent
\textbf{LTE-M/CAT-M1} je celulárny LPWAN protokol, ktorý pracuje na existujúcej LTE infraštruktúre s dátovými rýchlosťami až do 1 Mbps. LTE-M umožňuje komunikáciu v reálnom čase s nízkou latenciou, zaisťujúc spoľahlivú konektivitu pre mobilné zariadenia a nositeľnú elektroniku. Hoci LTE-M spotrebúva viac energie ako NB-IoT, ponúka vyvážený kompromis medzi výkonom a efektivitou pre časté prenosy dát. Ako sa rozširujú siete $5$G, LTE-M bude naďalej fungovať popri NB-IoT, profitujúc zo zlepšenej škálovateľnosti a pokrytia.

\begin{itemize}[$\bullet$]
    \item Typ: Celulárny (LTE); Rýchlosť: až 1 Mbps; Latencia: $< 60$ ms
    \item \textbf{Pozícia:} Rovnováha medzi rýchlosťou a úsporou energie. Podporovaná mobilita.
\end{itemize}

\vspace{0.5\baselineskip}
\subsubsection*{5. Weightless (SIG)}
\noindent
\textbf{Weightless} je otvorený štandard LPWAN navrhnutý pre IoT. Pracuje primárne v nelicencovaných sub-$1$ GHz frekvenciách a pristupuje k licencovanému spektru cez Weightless-P. Vyvinutý v partnerstve s ETSI a spravovaný neziskovou skupinou Weightless SIG, tento protokol ponúka flexibilné možnosti nasadenia a umožňuje rôznorodé prípady použitia vďaka svojmu otvorenému štandardu.

\begin{itemize}[$\bullet$]
    \item Typ: Otvorený štandard (\textit{open standard IoT LPWAN}), sub-1 GHz
    \item \textbf{Výhoda:} Transparentná štruktúra, prístup k štandardu, nekomerčné konzorcium.
\end{itemize}

\vspace{0.5\baselineskip}
\subsubsection*{6. Ingenu (RPMA – \textit{Random Phase Multiple Access})}
\noindent
\textbf{Ingenu (RPMA – Random Phase Multiple Access)} je proprietárna LPWAN technológia, ktorá pracuje v licencovanom spektre. Poskytuje konektivitu na veľké vzdialenosti s vysokou priepustnosťou, ponúkajúc robustný výkon siete, odolnosť voči rušeniu a bezpečnostné funkcie.

\begin{itemize}[$\bullet$]
    \item Typ: Licencované spektrum, proprietárna technológia
    \item \textbf{Oblasť:} Priemysel, kritická infraštruktúra, energetika.
\end{itemize}

\subsubsection{Prehľad Kľúčových Funkčných Rozdielov}
\label{sec:comparative_matrix_overview} % Zmenený label, aby sa predišlo duplicite

\noindent
Ako je vidieť z porovnávacej matice, LPWAN technológie sa delia na dve hlavné skupiny: \textbf{Celulárne (NB-IoT, LTE-M)} a \textbf{Necelulárne (LoRaWAN, Sigfox)}. Kľúčové zistenia:
\begin{itemize}
    \item \textbf{Rýchlosť a Latencia:} \textbf{LTE-M} je jasným lídrom, ponúka rýchlosti až $1$ Mbps a nízku latenciu ($< 60$ ms), čo ho robí ideálnym pre mobilitu a aplikácie v reálnom čase. NB-IoT ponúka miernu rýchlosť a vyššiu latenciu.
    \item \textbf{Dosah a Paket:} \textbf{Sigfox} a \textbf{LoRaWAN} dominujú v dosahu (až $40$ km a $20$ km) a sú vhodnejšie pre veľké oblasti, ale obmedzujú veľkosť paketu a majú vyššiu latenciu. Sigfox je najviac obmedzený v objeme dát ($12$ B).
    \item \textbf{Spektrum a Nasadenie:} Celulárne technológie pracujú v \textbf{licencovanom} spektre, čo garantuje QoS, ale vyžaduje operátora. LoRaWAN a Sigfox využívajú \textbf{nelicencované} spektrum, čo umožňuje vytváranie \textbf{súkromných sietí} (najmä LoRaWAN), čo je významná výhoda pre autonómne podniky.
    \item \textbf{Spotreba Energie:} Všetky LPWAN technológie poskytujú dlhú životnosť batérie (roky), čo potvrdzuje ich spoločný cieľ.
\end{itemize}

\begin{longtable}{l c c c c}
    \caption{Porovnávacia Matica Parametrov LPWAN Technológií}
    \cite{5gtechworld2024_lpwan}
    \label{tab:matrix_comparison} \\
    \toprule
    \textbf{Parameter} & \textbf{NB-IoT} & \textbf{LTE-M} & \textbf{LoRaWAN} & \textbf{Sigfox} \\
    \midrule
    \endfirsthead
    \caption{Pokračovanie: Porovnávacia Matica Parametrov LPWAN Technológií} \\
    \toprule
    \textbf{Parameter} & \textbf{NB-IoT} & \textbf{LTE-M} & \textbf{LoRaWAN} & \textbf{Sigfox} \\
    \midrule
    \endhead
    \textbf{Spektrum} & Licencované & Licencované & Nelicencované & Nelicencované \\
    \textbf{Dosah (mesto/vidiek)} & $\approx 1 / 10$ km & $\approx 1 / 10$ km & $\approx 5 / 20$ km & $\approx 10 / 40$ km \\
    \textbf{Rýchlosť} & $26-66$ kbps & až 1 Mbps & $0.3-5.5$ kbps & $0.1$ kbps \\
    \textbf{Životnosť batérie} & Roky & Roky & Roky & Roky \\
    \textbf{Latencia} & $1.2-10$ s & $< 60$ ms & Sekundy & Sekundy \\
    \textbf{Mobilita} & Obmedzená & Áno & Áno & Áno \\
    \textbf{Súkromné siete} & Nie & Nie & Áno & Nie \\
    \textbf{Max. Paket} & 1280 B & 1280 B & $11-242$ B & 12 B (UL) \\
    \textbf{Výkon} & $20-120$ mW & $60-200$ mW & $25-100$ mW & $20-100$ mW \\
    \bottomrule
\end{longtable}

\subsubsection{Dodatočne: Rozdiel medzi Wi-Fi HaLow a LPWAN \cite{dfrobot2025_lpwan}}
\label{sec:halow_modernity}
\noindent
Tabuľka nižšie ukazuje, že \textbf{Wi-Fi HaLow} je výhodne pozicovaný ako jedinečný most v ekosystéme IoT. Tým, že pracuje v pásme sub-$1$ GHz a ponúka vysokú rýchlosť ($78$ Mbps), poskytuje výrazne \textbf{lepšiu priepustnosť} pri nízkej spotrebe energie ako tradičné LPWAN protokoly (NB-IoT, LoRaWAN). Jeho schopnosť podporovať viac ako $8000$ zariadení na jeden prístupový bod poskytuje bezprecedentnú škálovateľnosť pre priemyselné a inteligentné mestá, čím vypĺňa medzeru medzi vysokovýkonným Wi-Fi a nízkou priepustnosťou klasického LPWAN.
\vspace{0.5\baselineskip} 


\begin{table}[htbp]
    \centering
    \caption{Kľúčové Parametre Wi-Fi HaLow vs. Ostatné LPWAN}
    \label{tab:halow_summary}
    \begin{tabular}{l l p{5cm}}
        \toprule
        \textbf{Kritérium} & \textbf{Wi-Fi HaLow} & \textbf{Porovnanie s LPWAN} \\
        \midrule
        Frekvenčný Rozsah & $< 1$ GHz ($900$ MHz) & Lepšia penetrácia cez steny a vegetáciu \\
        Prevádzkový Dosah & do $1$ km & Menej ako LoRa/Sigfox, ale viac ako štandardné Wi-Fi \\
        Rýchlosť & až $78$ Mbps (krátke vzdial.) & Výrazne vyššia ako LPWAN \\
        Pripojenia & $> 8000$ zariadení & Škálovateľnosť ako NB-IoT \\
        Latencia & niekoľko ms & Lepšia ako väčšina LPWAN \\
        Spotreba Energie & oveľa menej ako štandardné Wi-Fi & Bližšie k LPWAN \\
        Licencovanie & Nelicencované, otvorený štandard & Nevyžaduje operátorov \\
        Pozícia & ``Most'' medzi Wi-Fi a LPWAN & Optimálna rovnováha rýchlosť $\leftrightarrow$ energia \cite{dfrobot2025_lpwan} \cite{velosiot2023_lpwancases}  \\
        \bottomrule
    \end{tabular}
\end{table}

% --- ZAČIATOK KAPITOLY 2 ---
% Zodpovedňy: @thenereg na discord
% Výnimočnosti HaLow
% ------------------------------------------------------

\section{HaLow: špecifiká a výhody}
\label{sec:halow_features_advantages}
\textit{Autori: Oleh Kysil}

\subsection{Fyzická úroveň (L1)}
\label{sec:physical_layer_l1}

\subsubsection{Rádiofrekvencia}
Rádiofrekvencia je dôležitou súčasťou akejkoľvek bezdrôtovej technológie na prenos dát. Určuje fyzické vlastnosti (čím vyššia frekvencia, tým vyššia priepustnosť, ale takéto signály sa ľahšie blokujú okolím) aj právne vlastnosti (každá krajina má svoje obmedzenia na používanie rádiofrekvencií a niektorí telekomunikační operátori platia milióny eur za právo používať frekvencie \cite{sk_rf_auction}). Štandard 802.11ah predpokladá použitie nízkofrekvenčného ISM (priemyselného, vedeckého a lekárskeho) spektra, pričom konkrétnu frekvenciu prispôsobuje obmedzeniam konkrétnej krajiny (napríklad ~870 MHz v EÚ, ~920 MHz v Japonsku, ~902 MHz v USA \cite{the_things_network_ism_list}). To umožňuje vyrábať a používať zariadenia bez drahých licencií a obmedzení po celom svete. Rodina technológií 801.11b-bn (lepšie známa ako Wi-Fi) tiež využíva ISM frekvencie a má úspech po celom svete.

\subsubsection{Rýchlosť prenosu dát}
Na rozdiel od iných technológií LPWAN, o ktorých sa hovorí v tomto článku, HaLOW môže v jednej sieti podporovať nízke rýchlosti (okolo 1 Mbit za sekundu), stredné (do 20 Mbit za sekundu) aj vysoké (do 300 Mbit/s) \cite{Sun2013IEEE8A} pri náležitom vývoji technológie. To umožňuje zjednotiť infraštruktúru a poskytuje priamy prístup k rôznym typom zariadení. Napríklad pre prístup k sieti LoRaWAN pomocou napríklad WiFi kamery je potrebný sprostredkovateľ, ktorým je zvyčajne špecialna brana (gatewaz). Ak budú nízkoenergetické senzory, kamery aj používateľské zariadenia používať štandard HaLow, prispeje to k nižšiemu oneskoreniu, ľahšiemu vývoju a používaniu. Napríklad: kamera s rozpoznávaním tváre môže otvoriť vzdialené dvere autorizovanej osobe bez použitia sprostredkovateľa alebo dodatočného komunikačného modulu. Zaujímavé je aj použitie komunikácie HaLow ako backhaul pre existujúce siete LpWAN, pričom sa zachová nízka spotreba energie. Takéto zariadenie bude schopné agregovať údaje z tisícov klasických koncových bodov (napríklad senzorov) a prenášať ich na veľké vzdialenosti.

\subsection{Výššie urovne OSI modelu}
Na rozdiel od niektorých štandardov opísaných v tomto článku, 802.11ah pochádza z rodiny štandardov 802.11 (lepšie známych ako WiFi) a štandardne zahŕňa vyššie vrstvy modelu OSI a podporuje štandardný TCP/IP stack. To umožňuje ľahšiu integráciu zariadení tohto formátu s internetom a umožňuje používať už overené riešenia a modely na ochranu, vývoj a údržbu siete a zariadení. Napríklad štandard LoRa definuje iba fyzickú (L1) vrstvu OSI modelu bez šifrovania alebo IP adresovania. Vyššie vrstvy OSI modelu pre LoRa poskytuje štandard LoRaWan.
% --- ZAČIATOK KAPITOLY 3 ---
% Použitie vo svete @Άρςεν_Κόζακ ⚜ @ADAMEČKO
% kde je používané kde sa dá použiť
% ------------------------------------------------------

\section{Aplikácie LPWAN Technológií a Rola Wi-Fi HaLow}
\label{sec:applications_and_halow}

\subsection{Súčasné použitie technológie LPWAN vo svete}
\label{sec:current_usage_slovak}
\textit{Autori: Arsen Kozak, Adam Kšenzulák}

\noindent
Wi-Fi HaLow sa už posunul mimo laboratórií a aktívne sa testuje v reálnych projektoch na viacerých kontinentoch. Jeho aplikácia je najviac viditeľná v krajinách, ktoré aktívne rozvíjajú IoT infraštruktúru, inteligentné mestá a poľnohospodárstvo založené na autonómnych systémoch.
\cite{iotanalytics2018_lpwan}
\vspace{0.5\baselineskip}
\noindent
V \textbf{Číne} je Wi-Fi HaLow aktívne propagovaný v rámci programov ''Smart Agriculture'' a "New Industrial Internet". Konkrétne v provinciách Guangdong a Jiangsu sa implementujú systémy na monitorovanie stavu pôdy a klímy pomocou Wi-Fi HaLow senzorov. Spoločnosti ako \textbf{Huawei} a \textbf{Hikvision} ho integrujú do video monitorovacích zariadení pre veľké oblasti.
\cite{iotanalytics2018_lpwan}
\noindent
V \textbf{Austrálii} technológia získala trakciu v agro-priemyselnom komplexe. Spoločnosti ako \textbf{Morse Micro} sa zameriavajú na nasadzovanie sietí na veľkých farmách pre monitorovanie dobytka, kontrolu hladiny vody v nádržiach a požiarnu bezpečnosť, kde je pokrytie Wi-Fi, BLE a ZigBee nedostatočné.
\cite{iotanalytics2018_lpwan}
\noindent
V \textbf{Európe} (Nemecko, Slovensko, Holandsko) sa HaLow primárne používa v projektoch inteligentných miest na pripojenie LED pouličných svetiel a environmentálnych senzorov, poskytujúc vysokú rýchlosť pre prenos diagnostických dát a aktualizácií firmvéru na veľké vzdialenosti.
\cite{iotanalytics2018_lpwan}
\noindent
V \textbf{USA} sa Wi-Fi HaLow používa pre automatizáciu inteligentných domácností, najmä na rozšírenie dosahu IoT zariadení mimo domu na pokrytie veľkých pozemkov. Taktiež sa aplikuje v priemyselných parkoch na sledovanie majetku (asset tracking) a monitorovanie komplexných výrobných procesov, kde je potrebná vysoká spoľahlivosť a nízka latencia.

\subsubsection*{Kľúčové Trhy a Pozícia HaLow}
\begin{itemize}
    \item \textbf{Čína:} Inteligentné poľnohospodárstvo, priemyselné moduly (Huawei, Hikvision).
    \item \textbf{Austrália:} Farmy, požiarna bezpečnosť, monitorovanie nádrží (Morse Micro).
    \item \textbf{Európa:} Osvetlenie v Smart City, environmentálne senzory (Nemecko, Slovensko, Holandsko).
\end{itemize}
\noindent
\textbf{Myšlienka:} Nie nahradiť LPWAN, ale doplniť ho: vytvoriť univerzálny "most" pre systémy, ktoré vyžadujú obojsmernú komunikáciu, vyššiu rýchlosť a nižšiu spotrebu energie ako štandardné Wi-Fi. \cite{iotanalytics2018_lpwan}

\subsection{Trhové Trendy a Regionálna Vhodnosť LPWAN}
\label{sec:regional_trends}
\begin{table}[htbp]
    \centering
    \caption{Trhové Trendy a Regionálna Vhodnosť LPWAN~\cite{iotanalytics2018_lpwan}} 
    \label{tab:regional_trends}
    \begin{tabular}{l p{4cm} c p{5cm}}
        \toprule
        \textbf{Technológia} & \textbf{Hlavné Regióny} & \textbf{Podiel na Trhu} & \textbf{Trendy} \\
        \midrule
        NB-IoT & Čína (84\%), EÚ, Blízky Východ & $ \sim 20\%$ (2023) & Rozvoj smart-city, urban-IoT \\
        LTE-M & USA, EÚ, Austrália & $\sim 32\%$ (2023) & Rast vďaka LTE infraštruktúre \\
        LoRaWAN & Sev. Amerika, EÚ, APAC & $\sim 40\%$ (2023) & Líder mimo Číny, otvorený ekosystém \\
        Sigfox & EÚ (Francúzsko, Španielsko), Japonsko, Brazília & Menší, ale stabilný & Medzera pre nízkonákladové senzory \\
        \bottomrule
    \end{tabular}
\end{table}

\subsection{Hlavné Priemyselné Aplikácie}
\label{sec:industry_applications}
\begin{table}[htbp]
    \centering
    \caption{Hlavné Priemyselné Aplikácie LPWAN Technológií ~ \cite{5gtechworld2024_lpwan}}
    \label{tab:applications}
    \begin{tabular}{p{5cm} l p{5cm}}
        \toprule
        \textbf{Odvetvie} & \textbf{LPWAN Technológie} & \textbf{Príklady} \\
        \midrule
        Smart City / Mestský Život & LoRaWAN, NB-IoT, Sigfox & Parkovanie, osvetlenie, monitorovanie ovzdušia, odpad \\
        Maloobchod / Logistika & NB-IoT, LTE-M, LoRaWAN & Inteligentné police, RFID monitorovanie, správa zásob \\
        Priemyselný IoT / Automatizácia & LTE-M, NB-IoT, LoRaWAN & Výrobné linky, kontrola zariadení \\
        Inteligentné Poľnohospodárstvo / Životné Prostredie & LoRaWAN, Sigfox, NB-IoT & Vlhkosť pôdy, dobytok, kontrola požiarov \\
        eHealth / Zdravotníctvo & LTE-M, NB-IoT, LoRaWAN & Nositeľné zariadenia, monitorovanie pacientov, telemedicína \\
        \bottomrule
    \end{tabular}
\end{table}

\vspace{0.5\baselineskip}
\noindent
Analýza priemyselných aplikácií jasne ukazuje, že \textbf{žiadna jednotlivá LPWAN technológia nie je univerzálna}. Každý protokol obsadzuje svoju vlastnú medzeru na trhu:

\begin{itemize}
    \item \textbf{LoRaWAN a Sigfox} sú nenahraditeľné tam, kde sú kritické \textbf{maximálny dosah} a \textbf{minimálna spotreba energie} (poľnohospodárstvo, monitorovanie životného prostredia), aj keď za cenu nízkej rýchlosti a vysokej latencie.

    \item \textbf{NB-IoT a LTE-M} dominujú v mestských a priemyselných prostrediach, kde sú potrebné \textbf{spoľahlivosť, zaručená kvalita služby (QoS)} a vyššia úroveň mobility (inteligentné mestá, logistika).

    \item Nový hráč, \textbf{Wi-Fi HaLow}, sa pozicionuje ako \textbf{"Most"}, ponúkajúc výrazne vyššiu rýchlosť a nižšiu latenciu v porovnaní s ostatnými LPWAN, pričom si zachováva dobrú energetickú efektivitu a dosah potrebný pre podnikový a domáci IoT.
\end{itemize}

% návrhy toho, kde sa ešte dá použiť prínos do sveta
% @ADAMEČKO
% ------------------------------------------------------ %
% --- PERSPEKTÍVY ROZVOJA A VYUŽITIA V  BUDÚCNOSTI ---
\subsubsection{Perspektívy rozvoja a využitia LPWAN v budúcnosti}
\label{sec:perspektivy}
\textit{Autor: Adam Kšenzulák}

Budúcnosť rozsiahlych sietí s nízkou spotrebou energie (LPWAN) je neoddeliteľne spojená s masívnym nasadením \textbf{Internetu vecí (IoT)}. Očakáva sa, že žiadna jednotlivá technológia nebude dominovať, ale trh sa rozdelí podľa špecifických požiadaviek, pričom **Wi-Fi HaLow** je pripravený zohrať kľúčovú úlohu ako hybridné riešenie \cite{newracom_disrupt}.

\paragraph{1. Kľúčová pozícia Wi-Fi HaLow: Most medzi protokolmi}

Štandard \textbf{Wi-Fi HaLow (IEEE 802.11ah)} sa vyvíja, aby zaplnil medzeru medzi krátkodosahovým Wi-Fi a tradičnými LPWAN protokolmi. Jeho perspektíva spočíva v unikátnej kombinácii vlastností:

\begin{itemize}
    \item \textbf{Vysoká priepustnosť a efektivita:} HaLow ponúka dátové rýchlosti až do \textbf{78 Mbps} \cite{silex_technology_lpwan}, čo je rádovo stovkykrát viac ako pri LoRa/Sigfox \cite{newracom_disrupt}. Táto rýchlosť je kľúčová pre aplikácie vyžadujúce prenos väčších dátových paketov, obrazových súborov, alebo pre \textbf{rýchle aktualizácie firmvéru (FOTA)}. HaLow dosahuje energetickú efektívnosť tým, že vďaka rýchlejšiemu prenosu zostáva rádio aktívne len veľmi krátku dobu a rýchlo prechádza do režimu spánku \cite{newracom_disrupt}.
    \item \textbf{Rozšírený dosah a penetrácia:} Vďaka prevádzke vo frekvenčnom pásme pod 1 GHz dosahuje HaLow dosah až \textbf{1 km} v mestskom prostredí a poskytuje lepšiu penetráciu budovami a prekážkami ako štandardné Wi-Fi (2,4 GHz/5 GHz). To ho robí ideálnym pre rozsiahle interiérové a exteriérové nasadenia \cite{rcr_wireless_cases, mobile_world_live_halow}.
    \item \textbf{Škálovateľnosť a jednoduchosť integrácie:} HaLow využíva natívnu \textbf{IP konektivitu} a známy Wi-Fi ekosystém. Vďaka mechanizmu \textbf{Restricted Access Window (RAW)} je schopný pod jedným prístupovým bodom efektívne obslúžiť tisíce zariadení, čo je kľúčové pre veľké priemyselné a mestské senzorové siete \cite{mobile_world_live_halow}.
\end{itemize}

\paragraph{2. Dominantné aplikačné scenáre pre Wi-Fi HaLow}

HaLow má potenciál narušiť tradičné segmenty LPWAN a WPAN v nasledujúcich oblastiach:

\begin{itemize}
    \item \textbf{Priemyselný IoT (IIoT):} Spoľahlivé, rozsiahle pokrytie pre \textbf{sledovanie majetku v reálnom čase} (Asset Tracking) a \textbf{video dohľad} v skladoch a výrobných halách \cite{rcr_wireless_cases}.
    \item \textbf{Smart Budovy a Kampusy:} Ideálny pre komplexné riadenie budov (HVAC, energetický manažment) a pre rozsiahle senzorové siete na veľkých pozemkoch \cite{rcr_wireless_cases}.
    \item \textbf{Inteligentné Poľnohospodárstvo (Smart Farming):} Schopnosť pokryť rozsiahle polia jediným prístupovým bodom a poskytovať dostatočnú rýchlosť pre \textbf{monitoring a automatizáciu} \cite{rcr_wireless_cases}.
\end{itemize}

\paragraph{3. Celkový výhľad a konvergencia}

Hoci NB-IoT a LTE-M budú naďalej profitovať z licencovaného spektra a etablovaných celulárnych sietí, HaLow poskytuje silnú alternatívu, ktorá je cenovo efektívnejšia a technicky vhodnejšia pre aplikácie vyžadujúce \textbf{balans medzi dosahom, vysokou hustotou a priepustnosťou} v rámci existujúcej IP infraštruktúry \cite{mobile_world_live_halow}.

\section{Návrh experimentálneho výskumu}
\label{sec:vyskum}

Na základe analýzy teoretických vlastností technológie Wi-Fi HaLow a jej porovnania s konkurenčnými LPWAN štandardmi (kapitola \ref{sec:lpwan_vs_others}) bol formulovaný návrh experimentálneho výskumu. Cieľom nie je len implementácia technológie, ale predovšetkým empirická verifikácia jej prenosových parametrov v špecifických podmienkach, ktoré nie sú dostatočne pokryté existujúcou literatúrou.

\subsection{Vedecké ciele a hypotézy}
Hlavným cieľom výskumu je kvantifikovať vplyv fyzických prekážok a elektromagnetického rušenia na priepustnosť a latenciu siete IEEE 802.11ah v porovnaní so štandardom LoRaWAN.

\noindent Stanovujeme nasledujúce výskumné otázky a hypotézy:
\begin{itemize}
    \item \textbf{Hypotéza H1:} V prostredí s vysokou hustotou prekážok (železobetónové konštrukcie) vykáže Wi-Fi HaLow pokles priepustnosti o menej ako 40 \% oproti referenčnému meraniu v priamej viditeľnosti (LOS), zatiaľ čo latencia zostane pod hranicou 100 ms.
    \item \textbf{Hypotéza H2:} Energetická efektivita mechanizmu Target Wake Time (TWT) pri HaLow prekoná režim spánku LoRaWAN triedy A pri intervaloch odosielania dát kratších ako 60 sekúnd.
\end{itemize}

\subsection{Metodika výskumu (Priebeh)}
Výskum bude prebiehať v troch fázach, pričom kombinuje simulačné modelovanie s fyzickým meraním na reálnom hardvéri.

\subsubsection*{Fáza 1: Simulačné modelovanie}
V prvej fáze bude vytvorený topologický model siete v prostredí \textbf{NS-3 (Network Simulator 3)} s modulom pre 802.11ah.
\begin{itemize}
    \item \textbf{Simulačný scenár:} Hviezdicová topológia s jedným prístupovým bodom (AP) a variabilným počtom staníc (STA) od 10 do 1000.
    \item \textbf{Sledované parametre:} Bude sa simulovať zaťaženie siete (Network Load) a sledovať miera stratených paketov (Packet Error Rate - PER) pri kolíziách.
\end{itemize}

\subsubsection*{Fáza 2: Experimentálne meranie (Testbed)}
Pre fyzické merania bude zostavené testovacie prostredie pozostávajúce z modulov (napr. Newracom alebo Alfa Network) operujúcich vo frekvenčnom pásme 868 MHz (EU).
Merania budú prebiehať v dvoch prostrediach:
\begin{enumerate}
    \item \textbf{Laboratórne podmienky (Interiér):} Testovanie penetrácie signálu cez steny rôznej hrúbky (tehla, betón).
    \item \textbf{Exteriér (Areál STU):} Meranie maximálneho dosahu pri zachovaní minimálnej priepustnosti 150 kbps.
\end{enumerate}

\subsection{Metriky a zber dát}
Kľúčovým aspektom vedeckej časti je presný zber dát. Počas experimentov budú zaznamenávané nasledujúce metriky:
\begin{itemize}
    \item \textbf{RSSI (Received Signal Strength Indicator):} Úroveň prijatého signálu [dBm].
    \item \textbf{SNR (Signal-to-Noise Ratio):} Odstup signálu od šumu [dB].
    \item \textbf{Throughput:} Reálna dátová priepustnosť na aplikačnej vrstve (merané pomocou nástroja Iperf).
    \item \textbf{Latency:} Obojsmerné oneskorenie (RTT) pre ICMP pakety.
\end{itemize}

\section{Predpokladané výstupy a prínos výskumu}
\label{sec:vystupy}

Realizácia navrhovaného výskumu prinesie nasledujúce konkrétne výstupy, ktoré presahujú rámec bežnej implementácie:

\begin{enumerate}
    \item \textbf{Komparatívna matica výkonnosti:} Súbor nameraných dát porovnávajúci HaLow a LoRaWAN v identických podmienkach. Tento výstup potvrdí alebo vyvráti marketingové tvrdenia výrobcov čipov.
    \item \textbf{Model útlmu signálu:} Matematický model popisujúci šírenie signálu 802.11ah v špecifickom prostredí univerzitného kampusu, ktorý môže byť použitý pre plánovanie IoT sietí v podobných budovách.
    \item \textbf{Odporúčania pre konfiguráciu:} Sada optimálnych parametrov (MCS index, Guard Interval) pre rôzne scenáre nasadenia (statické senzory vs. video stream), overená experimentálne.
\end{enumerate}

Výsledky výskumu budú slúžiť ako podklad pre návrh robustnej senzorovej siete pre monitorovanie environmentálnych veličín v rámci "Smart Campus" riešení, pričom poskytnú vedecky podložené argumenty pre výber komunikačnej technológie.

\newpage
\bibliographystyle{unsrt}
\bibliography{literatura}

\end{document}
