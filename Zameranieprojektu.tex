% !TEX TS-program = pdflatex
% !TEX encoding = UTF-8 Unicode

\documentclass[11pt]{article}

\usepackage[utf8]{inputenc}

%DIMENZIE
\usepackage{geometry}
\geometry{a4paper} 

\usepackage{graphicx} 
\usepackage[IL2]{fontenc}
%PACKAGES
\usepackage{booktabs} 
\usepackage{array} 
\usepackage{paralist} 
\usepackage{verbatim} 
\usepackage{subfig} 

% HEADERS & FOOTERS
\usepackage{fancyhdr}
\pagestyle{fancy} 
\renewcommand{\headrulewidth}{0pt}
\lhead{}\chead{}\rhead{}
\lfoot{}\cfoot{\thepage}\rfoot{}

\usepackage{sectsty}
\allsectionsfont{\sffamily\mdseries\upshape}

\usepackage[nottoc,notlof,notlot]{tocbibind} 
\usepackage[titles,subfigure]{tocloft} 
\renewcommand{\cftsecfont}{\rmfamily\mdseries\upshape}
\renewcommand{\cftsecpagefont}{\rmfamily\mdseries\upshape}

\title{ \bf Výskum WiFi HaLow Technológie}
\author{Adam Kšenzulák\and Arsenii Leno\and Oleh Kysil\and Tobias Lačný}
\date{}

\begin{document}
\maketitle
\paragraph {\bf Akronym: \tt ReWiLow}
\section*{\bf Charakteristika projektu}

Naša skupina skúma štandard IEEE 802.11ah pre LPWAN (Low Power Wide Area Network – Nízkoenergetická rozsiahla sieť), bežne známy ako Wi-fi HaLow, so zameraním  na jeho architektonický dizajn, výkonové charakteristiky a vhodnosť pre komunikáciu v rámci Internetu vecí (IoT), komuníkaciu medzi strojmi (M2M) aj na trhu. Skúmané sú aj jedinečné technické vlastnosti HaLow, ako pre prevádzka pod 1GHz v nelicencovaných rádiových (ISM) pásiem, rozsšírený dosah, nízka spotreba energie, a aj vysoká hustota zariadení. Hodnotená je aj jeho použiteľnosť v reálnych scenároch, vrátane inteligentného poľnohospodárstva, priemyselnej automatizácie, monitorovania životného prostredia a inteligentných miest.
V rámci projektu sa bude zameriavať aj na komparatívnu analýzu medzi Wi-Fi HaLow a inými LPWAN technológiami, príkladom sú LoRaWAN, Sigfox, alebo aj NB IoT najmä v ich parametroch, ako sú dosah, rýchlosť prenosu dát, latencia, energetická účinnosť a škálovateľnosť siete. Našim cieľom bude určiť konkrétne kontexty, v ktorých Wi-Fi HaLow ponúka výhody alebo obmedzenia v porovnaní s konkurečnými riešeniami LPWAN. Výsledky majú slúžiť ako podklad pre budúci rozvoj infraštruktúry IoT a stratégií konektivity na Slovensku

\end{document}
