%%%TOTO JE FORMULAR CELEHO REWILOW.

% !TEX TS-program = pdflatex
% !TEX encoding = UTF-8 Unicode

\documentclass[11pt]{article}

\usepackage[utf8]{inputenc}
\usepackage{amsmath}

%LANGUAGE
\usepackage[T1]{fontenc}
\usepackage[utf8]{inputenc}   % UTF-8
\usepackage[slovak]{babel}    % pre slovenský jazyk

%DIMENZIE
\usepackage{geometry}
\geometry{a4paper}
\usepackage{geometry}
\geometry{
    a4paper,
    top=2.5cm,      % Pole hore
    bottom=2cm,     % Pole dole
    left=3cm,       % Ľavé pole
    right=1.5cm,    % Prave pole
    headheight=14pt % Výška colontitula 
}

\usepackage{graphicx} 
\usepackage[IL2]{fontenc}
\usepackage{booktabs} 
\usepackage{array} 
\usepackage{paralist} 
\usepackage{verbatim} 
\usepackage{subfig} 

% HEADERS & FOOTERS
\usepackage{fancyhdr}
\pagestyle{fancy} 
\renewcommand{\headrulewidth}{0pt}
\lhead{}\chead{}\rhead{}
\lfoot{}\cfoot{\thepage}\rfoot{}

\usepackage{sectsty}
\allsectionsfont{\sffamily\mdseries\upshape}

\usepackage[nottoc,notlof,notlot]{tocbibind} 
\usepackage[titles,subfigure]{tocloft} 
\renewcommand{\cftsecfont}{\rmfamily\mdseries\upshape}
\renewcommand{\cftsecpagefont}{\rmfamily\mdseries\upshape}

\title{ \bf Výskum WiFi HaLow Technológie - Formulár}
\author{Adam Kšenzulák\and Arsenii Leno\and Oleh Kysil\and Tobias Lačný}
\date{}

\begin{document}
\maketitle


\section{\bf Formulár VV-A}

\subsection*{\bf A1}
\subsection{Evidenčné číslo projektu}
\qquad 01
\subsection{Dátum podania}
\qquad 5. novembra 2025
\subsection*{\bf A2}
\subsection*{\bf A2}
\begin{table}[!h]
\centering
\caption{Základné informácie o riešiteľskej organizácii / Basic Information on Participating Organization}
\label{tab:organizacia}
\begin{tabularx}{\linewidth}{|>{\hsize=0.08\hsize}X|>{\hsize=0.42\hsize}X|>{\hsize=0.5\hsize}X|}
\hline
\textbf{} & \textbf{Položka (SVK / ENG)} & \textbf{Vyplnená hodnota (SVK / ENG)} \\
\hline
\multicolumn{3}{|l|}{\textbf{Žiadateľ / Applicant}} \\
\hline
01 & \textbf{Názov organizácie} / \textbf{Name of the organization} & Slovenská technická univerzita v Bratislave – Fakulta informatiky a informačných technológií \\
\hline
\subsection*{\bf A3}
\subsection*{\bf A4}

\section{\bf Formulár VV-B}

% ===================================================================
% BLOK PRE FORMULÁR VV-B (štýl podľa kolegu)
% Vložte toto do vášho .tex súboru
% ===================================================================
\section{\bf Formulár VV-B}

\begin{table}[!h]
\centering
\caption{Ciele, zámery a výstupy projektu / Project objectives, aims and outputs}
\label{tab:vv-b_ciele}
\begin{tabularx}{\linewidth}{|l|l|X|}
\hline
\textbf{VV-B} & \textbf{Položka} & \textbf{Ciele, zámery a výstupy projektu} \\
\hline
01 & Kľúčové slová & WiFi HaLow, IEEE 802.11ah, LPWAN, IoT (Internet vecí), M2M, LoRaWAN, Sigfox, NB-IoT, Komparatívna analýza, Architektúra siete. \\
\hline
02 & Ciele projektu & 
Hlavným cieľom projektu je uskutočniť komplexnú teoretickú a komparatívnu analýzu štandardu IEEE 802.11ah (Wi-Fi HaLow) so zameraním na jeho architektonický dizajn, výkonové charakteristiky a vhodnosť pre IoT. Cieľom je určiť špecifické kontexty, v ktorých Wi-Fi HaLow ponúka výhody alebo obmedzenia v porovnaní s konkurenčnými LPWAN technológiami (LoRaWAN, Sigfox, NB-IoT). Výsledky majú slúžiť ako teoretický podklad pre budúce stratégie konektivity na Slovensku. \\
\hline
03 & TRL stupnica & TRL 2 \\
\hline
03 & Zdôvodnenie TRL & Projekt začína na úrovni TRL 2, keďže sa zameriava na formulovanie technologického konceptu a aplikačných možností Wi-Fi HaLow. \\
\hline
03 & Zaradenie výstupov projektu & 
Projekt dosiahne \textbf{TRL 3} (Analytické potvrdenie koncepcie), keďže jeho výstupy analyticky preukážu, pre ktoré aplikácie je HaLow životaschopný. \newline
\textbf{Popis výstupov:}
\begin{enumerate}
  \item \textbf{Výstup 1:} Komparatívna analytická štúdia (TRL 3) – hodnotí architektonický dizajn a výkonové charakteristiky HaLow v porovnaní s LoRaWAN a NB-IoT.
  \item \textbf{Výstup 2:} Súbor odporúčaní pre aplikačné scenáre (TRL 3) – definovanie špecifických kontextov (napr. smart poľnohospodárstvo, priemyselná automatizácia), kde HaLow poskytuje preukázateľné výhody.
\end{enumerate} \\
\hline
04 & Využitie výsledkov riešenia v praxi (žiadateľ) & 
\textbf{Áno.} Výsledky analýzy budú využité žiadateľom (STU FIIT) pre ďalší nadväzujúci výskum a ako podklad pre výučbu v predmetoch zameraných na IoT a bezdrôtové siete. \\
\hline
04 & Využitie výsledkov riešenia v praxi (iný odberateľ) & 
\textbf{Nie.} Keďže ide o študentský teoretický projekt, primárnym odberateľom je akademická pôda. \\
\hline
05 & Plánované výstupy a prínosy & 
\begin{enumerate}
  \item Prezentácia na Študentskej vedeckej a odbornej činnosti (ŠVOČ).
  \item Publikácia v zborníku z konferencie ŠVOČ.
  \item Finálna výskumná správa dostupná v akademickej knižnici STU.
\end{enumerate} \\
\hline
\end{tabularx}
\end{table}

\end{document}
