%%%TOTO JE FORMULAR CELEHO REWILOW.

% !TEX TS-program = pdflatex
% !TEX encoding = UTF-8 Unicode

\documentclass[11pt]{article}

\usepackage{amsmath}

%LANGUAGE
\usepackage[T1]{fontenc}
\usepackage[utf8]{inputenc}   % UTF-8
\usepackage[slovak]{babel}    % pre slovenský jazyk

%DIMENZIE
\usepackage{geometry}
\geometry{a4paper}
\usepackage{geometry}
\geometry{
    a4paper,
    top=2.5cm,      % Pole hore
    bottom=2cm,     % Pole dole
    left=3cm,       % Ľavé pole
    right=1.5cm,    % Prave pole
    headheight=14pt % Výška colontitula 
}

\usepackage{graphicx} 
\usepackage[IL2]{fontenc}
\usepackage{booktabs} 
\usepackage{tabularx}
\usepackage{array} 
\usepackage{paralist} 
\usepackage{verbatim} 
\usepackage{subfig} 
\usepackage{float}

% HEADERS & FOOTERS
\usepackage{fancyhdr}
\pagestyle{fancy} 
\renewcommand{\headrulewidth}{0pt}
\lhead{}\chead{}\rhead{}
\lfoot{}\cfoot{\thepage}\rfoot{}

\usepackage{sectsty}
\allsectionsfont{\sffamily\mdseries\upshape}

\usepackage[nottoc,notlof,notlot]{tocbibind} 
\usepackage[titles,subfigure]{tocloft} 
\renewcommand{\cftsecfont}{\rmfamily\mdseries\upshape}
\renewcommand{\cftsecpagefont}{\rmfamily\mdseries\upshape}

\title{ \bf Výskum WiFi HaLow Technológie - Formulár}
\author{Adam Kšenzulák\and Arsenii Leno\and Oleh Kysil\and Tobias Lačný}
\date{}

\begin{document}
\maketitle


\section{\bf Formulár VV-A}
\subsection*{\bf A1}  %TOBIAS
\begin{table}[H]
\centering
\caption{Zákkladné informacie o projekte}
\label {tab:basicinfo}
\begin{tabularx}{\linewidth}{|l|l|X|}
\hline
01 & Evidenčné číslo projektu & NIČ \\
\hline
02 & Dátum podania & 5. novembra 2025 \\
\hline
03 & Názov projektu & Výskum Wi-fi  HaLow technológie \\
\hline
04 & Akronym projektu & ReWiLow \\
\hline
05 & Odbor vedy a techniky & Prírodné vedy - Informatické vedy \\
\hline
06 & Charakter výskumu & Hodnotiaci aplikovaný výskum \\
\hline
07 & Začiatok riešenia projektu & 30. septembra 2025 \\
\hline
08 & Koniec riešenia projektu & 15. augusta 2028 \\
\hline
09 & Anotácia & Naša skupina skúma štandard IEEE 802.11ah pre LPWAN (Low Power Wide Area Network – Nízkoenergetická rozsiahla sieť), bežne známy ako Wi-fi HaLow, so zameraním  na jeho architektonický dizajn, výkonové charakteristiky a vhodnosť pre komunikáciu v rámci Internetu vecí (IoT), komuníkaciu medzi strojmi (M2M) aj na trhu. Skúmané sú aj jedinečné technické vlastnosti HaLow, ako pre prevádzka pod 1GHz v nelicencovaných rádiových (ISM) pásiem, rozsšírený dosah, nízka spotreba energie, a aj vysoká hustota zariadení. Hodnotená je aj jeho použiteľnosť v reálnych scenároch, vrátane inteligentného poľnohospodárstva, priemyselnej automatizácie, monitorovania životného prostredia a inteligentných miest.
V rámci projektu sa bude zameriavať aj na komparatívnu analýzu medzi Wi-Fi HaLow a inými LPWAN technológiami, príkladom sú LoRaWAN, Sigfox, alebo aj NB IoT najmä v ich parametroch, ako sú dosah, rýchlosť prenosu dát, latencia, energetická účinnosť a škálovateľnosť siete. Našim cieľom bude určiť konkrétne kontexty, v ktorých Wi-Fi HaLow ponúka výhody alebo obmedzenia v porovnaní s konkurečnými riešeniami LPWAN. Výsledky majú slúžiť ako podklad pre budúci rozvoj infraštruktúry IoT a stratégií konektivity na Slovensku \\
\hline
10 & Žiadateľská organizácia & STU FIIT \\
\hline
11 & Požadované finančné prostriedky & 300 000 € \\
\hline
12 & Spolufinancovanie projektu & --- \\
\hline
13 & Celkové náklady na projekt & približne 290 000 € \\
\hline

\end{tabularx}
\end{table}

\subsection*{\bf A2} %ADAM
\begin{table}[H]
\centering
\caption{Základné informácie o riešiteľskej organizácii / Basic Information on Participating Organization}
\label{tab:organizacia}
\begin{tabularx}{\linewidth}{|l|X|}
\hline
\textbf{Položka (SVK / ENG)} & \textbf{Vyplnená hodnota (SVK / ENG)} \\
\hline
Názov organizácie / Name of the organization & Slovenská technická univerzita v Bratislave – Fakulta informatiky a informačných technológií (STU FIIT) \\
\hline
Adresa organizácie / Address & Ilkovičova 2, 842 16 Bratislava 4 \\
\hline
IČO / Company ID & 00397687 \\
\hline
Príslušnosť k rezortu / Sector & Školstvo (Verejná vysoká škola) \\
\hline
Kontaktná osoba / Contact person & doc. Ing. Peter Trúchly, PhD. \\
\hline
Telefón / Phone & +421 2 21022531 \\
\hline
E-mail & truchly@fiit.stuba.sk \\
\hline
Štatutárny zástupca I & prof. Ing. Maximilián Strémy, PhD. (Rektor STU) \\
\hline
Štatutárny zástupca II & prof. Ing. Ivan Kotuliak, PhD. (Dekan FIIT STU) \\
\hline
\end{tabularx}
\end{table}

\subsection*{\bf A3} %ADAM
\begin{table}[H]
\centering
\caption{Riešiteľský tím / Research Team}
\begin{tabularx}{\linewidth}{|X|X|X|X|X|X|}
\hline
Meno a priezvisko & Pracovné zaradenie & Dátum narodenia & IČO organizácie & Počet hodín (3 roky) & Počet hodín (ročne) \\
\hline
Peter Trúchly, doc. Ing., PhD. & Docent / Associate Professor & 04. 06. 1975 & 00397687 & 510 & 170 \\
Adam Kšenzulák & Študent / Student & 07. 08. 2006 & 00397687 & 1890 & 630 \\
Arsenii Leno & Študent / Student & 26. 12. 2008 & 00397687 & 1890 & 630 \\
Oleh Kysil & Študent / Student & 23. 05. 2006 & 00397687 & 1890 & 630 \\
Tobias Lačný & Študent / Student & 23. 11. 2005 & 00397687 & 1890 & 630 \\
\hline
\end{tabularx}
\end{table}

\subsection*{\bf A4} %ADAM
\begin{table}[H]
\centering
\caption{Zodpovedný riešiteľ / Principal Investigator}
\begin{tabularx}{\linewidth}{|l|X|}
\hline
Položka & Hodnota \\
\hline
Meno a priezvisko & doc. Ing. Peter Trúchly, PhD. \\
\hline
Pohlavie & Muž \\
\hline
Dátum narodenia & 04. 06. 1975 \\
\hline
Telefón & +421 2 21022531 \\
\hline
Email & truchly@fiit.stuba.sk \\
\hline
Mladý vedecký pracovník & NIE \\
\hline
Typ vedeckej databázy & Web of Science (WoS), Scopus, ORCID \\
\hline
ID výskumníka & (Doplniť príslušné ID pre WoS, Scopus, ORCID) \\
\hline
Prehľad projektov (posledných 5 rokov) & 
1. Automotive innovation lab (KEGA č. 025STU-4/2022, 2022 – 2023, Vedúci projektu) \newline
2. Výskum WiFi HaLow Technológie (ReWiLow) (Interný/študentský výskumný projekt) \\
\hline
Počet projektov & 2 \\
\hline
Expertízy, konzultácie a iné & 
Expertíza pre partnera v oblasti komunikačných protokolov pre inteligentné vozidlá (V2X) a budovanie špecializovaného laboratória. \\
\hline
Počet expertíz & 1 \\
\hline
Aplikačné výstupy - chránené & 
Prototyp simulačného prostredia pre siete automobilovej komunikácie (TRL 4-5). (Nechránený výstup) \\
\hline
Počet aplikačných výstupov & 1 \\
\hline
\end{tabularx}
\end{table}

\section{\bf Formulár VV-B} %ARSENII
\begin{table}[H]
\centering
\caption{Ciele, zámery a výstupy projektu / Project objectives, aims and outputs}
\label{tab:vv-b_ciele}
\begin{tabularx}{\linewidth}{|l|l|X|}
\hline
\textbf{VV-B} & \textbf{Položka} & \textbf{Ciele, zámery a výstupy projektu} \\
\hline
01 & Kľúčové slová & WiFi HaLow, IEEE 802.11ah, LPWAN, IoT (Internet vecí), M2M, LoRaWAN, Sigfox, NB-IoT, Komparatívna analýza, Architektúra siete. \\
\hline
02 & Ciele projektu & 
Hlavným cieľom projektu je uskutočniť komplexnú teoretickú a komparatívnu analýzu štandardu IEEE 802.11ah (Wi-Fi HaLow) so zameraním na jeho architektonický dizajn, výkonové charakteristiky a vhodnosť pre IoT. Cieľom je určiť špecifické kontexty, v ktorých Wi-Fi HaLow ponúka výhody alebo obmedzenia v porovnaní s konkurenčnými LPWAN technológiami (LoRaWAN, Sigfox, NB-IoT). Výsledky majú slúžiť ako teoretický podklad pre budúce stratégie konektivity na Slovensku. \\
\hline
03 & TRL stupnica & TRL 2 \\
\hline
03 & Zdôvodnenie TRL & Projekt začína na úrovni TRL 2, keďže sa zameriava na formulovanie technologického konceptu a aplikačných možností Wi-Fi HaLow. \\
\hline
03 & Zaradenie výstupov projektu & 
Projekt dosiahne \textbf{TRL 3} (Analytické potvrdenie koncepcie), keďže jeho výstupy analyticky preukážu, pre ktoré aplikácie je HaLow životaschopný. \newline
\textbf{Popis výstupov:}
\begin{enumerate}
  \item \textbf{Výstup 1:} Komparatívna analytická štúdia (TRL 3) – hodnotí architektonický dizajn a výkonové charakteristiky HaLow v porovnaní s LoRaWAN a NB-IoT.
  \item \textbf{Výstup 2:} Súbor odporúčaní pre aplikačné scenáre (TRL 3) – definovanie špecifických kontextov (napr. smart poľnohospodárstvo, priemyselná automatizácia), kde HaLow poskytuje preukázateľné výhody.
\end{enumerate} \\
\hline
\end{tabularx}
\end{table}
\begin{table}[H]
\begin{tabularx}{\linewidth}{|l|l|X|}
\hline
04 & Využitie výsledkov riešenia v praxi (žiadateľ) & 
\textbf{Áno.} Výsledky analýzy budú využité žiadateľom (STU FIIT) pre ďalší nadväzujúci výskum a ako podklad pre výučbu v predmetoch zameraných na IoT a bezdrôtové siete. \\
\hline
04 & Využitie výsledkov riešenia v praxi (iný odberateľ) & 
\textbf{Nie.} Keďže ide o študentský teoretický projekt, primárnym odberateľom je akademická pôda. \\
\hline
05 & Plánované výstupy a prínosy & 
\begin{enumerate}
  \item Prezentácia na Študentskej vedeckej a odbornej činnosti (ŠVOČ).
  \item Publikácia v zborníku z konferencie ŠVOČ.
  \item Finálna výskumná správa dostupná v akademickej knižnici STU.
\end{enumerate} \\
\hline
\end{tabularx}
\end{table}

\subsection{\bf Formulár VV-C} %TOBIAS A ADAM
\begin{table}[H]
\centering
\caption{Rozpočet projektu}
\label{tab:vv-c Rozpocet}
\begin{tabularx}{\linewidth}{|X|c|X|X|X|X|X}
\hline
& Etapa (Roky) & 1. fáza (2025-2026) & 2. fáza (2026-2027) & 3. fáza (2027-2028)  & Celkovo:  \\
\hline
01 & Bežné priame náklady & -- & -- & -- & -- \\
\hline
02 & Mzdové a ostatné osobné náklady & 28 000 & 60 000 & 22 000 & 110 000 \\
\hline
03 & Zdravotné a sociálne poistenie & 9 000 & 20 000 & 7 000 & 36 000 \\
\hline
04 & Cestovné náklady & 4 000 & 7 000 & 6 000 & 17 000  \\
\hline
05 & Materiál & 12 000 & 35 000 & 6 000 & 53 000  \\
\hline
06 & Služby & 5 000 & 10 000 & 5 000 & 20 000\\
\hline
07 & Energie, vodné, stočné komunikácie & 3 000 & 5 000 & 2 000 & 10 000\\
\hline
08 & Náklady na popularizáciu & - & 3 000 & 10 000 & 13 000\\
\hline
09 & Bežné nepriame náklady & 9 000 & 15 000 & 7 000 & 31 000\\
\hline
10 & Bežné náklady spolu & 70 000 & 155 000 & 65 000 & 290 000 \\
\hline
Celkove náklady z APVV& & - & - & - & 300 000 \\ %Celkove naklady apvv
\hline
Celkové náklady &  & - & - & - & -\\ % Celkove naklady na rok
\hline
\end{tabularx}
\end{table}


\subsection{\bf Formulár VV-D} %TOBIAS
\begin{table}[H]
\centering
\caption{Harmonogram projektu}
\label{tab:vv-d harmonogram}
\begin{tabularx}{\linewidth}{|l|l|X|}
\hline
Začiatok etapy & Koniec etapy & Názov etapy \\
\hline %DÁTUMY ZAČÍNAJÚ
& & \\
\hline
30.9.2025 & 13.11.2025 & Školenie, štúdium literatúry \\
\hline
15.11.2025 & 31.12.2025 & Analýza trhu, tvorba metód riešenia \\
\hline
1.1.2026 & 14.2.2026 & Interné diskusie, meetingy ohľadne problematiky \\
\hline
15.2.2026 & 15.3.2026  & Nakúpenie prvotných produktov HaLow\\
\hline
16.3.2026 & 18.4.2026  & Vypracovanie prehľadového článku pre projekt\\
\hline
20.4.2026 & 16.5.2026  & Stretnutia s APVV ohľadne financovania\\
\hline
22.5.2026 & 15.6.2026  & Kontaktovanie univerzít, spoluriešiteľov, partnerov z praxe\\
\hline
& & \\
\hline
1.7.2026 & 31.8.2026 & Nákup HaLow technológií \\
\hline
1.9.2026 & 31.10.2026 & Zostrojenie scenárov, protokolov na testovanie HaLow Technológie\\
\hline
1.11.2026 & 30.11.2026 & Príprava simulácií, nastavenie sietí\\
\hline
1.12.2026 & 31.1.2027  & Komparatívna analýza s inými produktami, rovnako testované\\
\hline
1.2.2027 & 31.3.2027  & Testovanie produktov HaLow\\
\hline
1.4.2027 & 14.5.2027  & Zber dát, štatistické spracovanie priebežných výsledkov\\
\hline
20.5.2027 & 15.6.2027  & Prezentácie priebežných výsledkov a štatistík pre APVV \\
\hline
& & \\
\hline
1.7.2027 & 31.8.2027 & Posledné experimenty, overovania experimentov \\
\hline
1.9.2027 & 30.9.2027  & Analýza nazbieraných dát\\
\hline
1.10.2027 & 31.10.2027  & Finalizácia, prepis článkov\\
\hline
1.11.2027 & 20.12.2027  & Metodológia, dotestovanie originálnych využití HaLow\\
\hline
30.12.2027 & 30.1.2028  & Dohoda a publikovanie vo vedeckých článkoch\\
\hline
1.2.2028 & 1.6.2028 & Verejné konferencie, prednášky na univerzitách, mediálne videá\\
\hline
1.5.2028 & 15.6 2028  & Odporučania pre budúcnosť výskumu HaLow technológie\\
\hline
20.6.2028 & 15.8.2028 & Stretnutie s APVV - Ukončenie projektu \\
\hline

\end{tabularx}
\end{table}

\subsection{\bf Formulár VV-F}
\begin{table}[H]
\begin{tabularx}{\linewidth}{|X|X|}
\hline
Názov projektu & Výskum Wi-Fi HaLow technológie \\
\hline
Zodpovedný riešiteľ & doc. Ing. Peter Trúchly, PhD. \\
\hline
Žiadateľ & STU FIIT \\
\hline
Štatutárny/i zástupca/ovia žiadateľa & prof. Ing. Ivan Kotuliak, PhD. (Dekan FIIT STU) \\
\hline
\end{tabularx}
\end{table}

\subsubsection{Excelentnosť}

Predkladaný projekt si kladie za cieľ komplexnú analýzu a detailné porovnanie inovatívnej bezdrôtovej technológie LPWAP 802.11ah, ktorá je všeobecne známa aj pod obchodným názvom HaLow. Keďže táto pokročilá technológia je v súčasnosti na trhu zatiaľ málo rozšírená a jej nasadenie nie je príliš frekventované, tento projekt sa prioritne zameriava aj na dôkladné preskúmanie a následné objasnenie všetkých reálnych možností a podmienok potrebných pre úspešné vytvorenie a spustenie verejnej siete HaLow na území Slovenskej republiky. Pre dosiahnutie týchto cieľov sa v rámci tohto projektu systematicky využívajú a syntetizujú už existujúce výsledky a poznatky z relevantných vedeckých článkov a štúdií. Tieto články obsahujú rozsiahle dáta z meraní charakteristík a testov, ktoré detailne popisujú možnosti tohto štandardu tak v teoretickej rovine, prostredníctvom modelových simulácií, ako aj v praktickej podobe, kde boli vykonané testy s reálnym, komerčne dostupným hardvérom a softvérom.

Hlavným zámerom a primárnym cieľom tohto projektu je preto hlboké štúdium všetkých dostupných existujúcich článkov a publikácií, ktoré sa venujú téme štandardu HaLow. Popri tom sa bude klásť značný dôraz na skúmanie a identifikáciu nových potenciálnych možností použitia tejto technológie v kontexte každodenného života, aby sa Maximalizoval jej praktický prínos. Súčasne sa bude venovať pozornosť realizácii a možnostiam úspešného rozvinutia robustnej a spoľahlivej siete HaLow v špecifických podmienkach Slovenska. Samotná realizácia tohto projektu bude prebiehať dvojakým spôsobom. Na jednej strane sa budú využívať moderné virtuálne simulátory, ktoré umožnia efektívne modelovanie a predbežné testovanie rôznych scenárov a konfigurácií. Na druhej strane sa uskutočnia aj fyzické testy funkčnosti malej, pilotnej siete, aby sa získali reálne dáta a overila sa praktická aplikovateľnosť teoretických poznatkov. V nadväznosti na detailné vyhodnotenie výsledkov získaných z týchto testov, či už virtuálnych alebo fyzických, existuje potenciál, že poskytnuté technické vybavenie, ktoré bude použité pri realizácii, môže slúžiť ako pevný a spoľahlivý základ pre budovanie oveľa rozsiahlejšej a komplexnejšej siete HaLow. Takýto prístup by mohol výrazne prispieť k zníženiu celkových nákladov a zároveň k minimalizovaniu technickej zložitosti experimentovania a ďalšieho rozvoja s touto technológiou pre všetkých potenciálnych záujemcov a operátorov.

Na tomto projekte budú primárne participovať mladší zamestnanci, konkrétne tí vo veku do 35 rokov. Táto strategická voľba personálu je odôvodnená hlavným dôrazom, ktorý je v rámci projektu kladený na vyššie úrovne referenčného modelu OSI (Open Systems Interconnection). Tieto úrovne, ako napríklad prezentačná alebo aplikačná vrstva, typicky nevyžadujú extrémne rozsiahle a špecifické znalosti v oblasti rádiovej techniky alebo hlbšieho pochopenia fyzických aspektov prenosu dát. Avšak, starší a skúsenejší zamestnanci, disponujúci bohatými odbornými znalosťami, môžu byť do projektu zapojení, ak si situácia vyžiada špecializovanú expertízu. Ich zapojenie by bolo obzvlášť prínosné v prípadoch, keď bude potrebná dôkladná odborná znalosť v oblasti rádiovej techniky, teda na úrovni L1 (fyzická vrstva) modelu OSI, ktorá sa zaoberá základnými fyzickými a elektrickými špecifikáciami.

\subsubsection{Dopad}

Hlavnými a najvýznamnejšími produktmi, ktoré vyplynú z tohto ambiciózneho projektu, sú predovšetkým pokus o reálne založenie a následné úspešné spustenie funkčnej siete HaLow na území Slovenskej republiky. Okrem toho sa projekt sústredí na rozsiahly výskum tejto prelomovej technológie, vrátane jej aktuálneho stavu vývoja, detailnej analýzy nových potenciálnych aplikácií a dôkladného zmapovania jej súčasného stavu implementácie a rozšírenia priamo na Slovensku. Výsledky a závery, ktoré projekt prinesie, budú mať slúžiť ako cenný referenčný model a inšpiratívny príklad pre ďalšie mestá a obce nielen na Slovensku, ale aj v susedných krajinách. Projekt sa intenzívne zameriava na systematické hľadanie a objavovanie optimálnych materiálnych, čiže hardvérových, a taktiež softvérových riešení, ktoré sú nevyhnutné pre úspešné a efektívne rozvinutie spoločnej siete HaLow. Tento prístup prinesie značnú úsporu času a finančných prostriedkov pre budúcich prevádzkovateľov tejto technológie po celom svete, keďže budú môcť využiť overené postupy a riešenia.

Vzhľadom na strategickú geografickú polohu Bratislavy, kde sa plánuje realizovať podstatná, hlavná časť projektu, v tesnej blízkosti hraníc s Rakúskom a Maďarskom, ako aj s prihliadnutím na dlhý dosah a rozsiahle pokrytie, ktoré táto diskutovaná technológia ponúka, existuje značná možnosť zapojiť do projektu aj medzinárodných partnerov pochádzajúcich z týchto susedných krajín. Tento projekt poskytne Slovensku jedinečnú príležitosť rýchlejšie sa adaptovať a integrovať novú generáciu technológií. Táto rýchla adaptácia bude mať priamy a veľmi pozitívny vplyv na celkovú pridanú hodnotu produktov a služieb, ktoré sú vyrábané a poskytované v Slovenskej republike. Okrem ekonomického prínosu to tiež povedie k podstatne efektívnejšiemu využívaniu prírodných zdrojov, čím sa dosiahne ich výrazná úspora a zároveň sa minimalizuje negatívny dopad na životné prostredie. Konkrétnym príkladom implementácie takýchto sietí môže byť ich využitie na strategické umiestňovanie kamier v rozsiahlych lesných oblastiach. Tieto kamery by slúžili na kontinuálne monitorovanie a sledovanie nelegálnej ťažby dreva, čím by prispievali k ochrane nášho prírodného bohatstva. Rovnako by mohli byť nasadené pre potreby biologického výskumu, umožňujúc získavanie cenných dát v reálnom čase priamo z terénu. Náš projektový tím je pevne presvedčený o tom, že grantové prostriedky budú využité s maximálnou efektivitou. Toto presvedčenie je podložené predovšetkým možnosťou efektívneho využitia už existujúcej infraštruktúry Slovenskej technickej univerzity (STU), čo eliminuje potrebu investícií do novej základnej výbavy. Navyše, sú k dispozícii relatívne nízke náklady na technické vybavenie, ktoré je nevyhnutné pre úspešnú realizáciu fyzickej časti tejto rozsiahlej a inovatívnej práce.

\subsubsection{Implementácia}

\begin{itemize}
\item { Projekt je rozdelený do troch hlavných fáz, ktoré sú popísané vo formulári VV-D aj s predpokládanými trvaniami daných podetáp\\
Každá fáza je rozdelená do menších pracovných balíkov (skrátene PB), v ktorých sú uvedené aj milníky výskumu, na konci popisu fáze je jej výstup pre celkový projekt
\begin{itemize}
\item {\textbf{1. Fáza - Analýza literatúry a návrh plánu skúmania:}\\
 {Cielom prvej fázy je získať poznatky o skúmanom štandarde 802.11ah (alebo Wi-fi HaLow) a vypracovanie plánu experimentov.}
\begin{itemize}
\item \textbf{PB1} - Štúdium literatúry, analýza trhu\\
Prehľad vedeckých článkov o štandarde a LPWAN technológiách

\textbf{Milník č.1: Vypracovanie analytickej správy o štandarde Wi-fi HaLow a konkurenčnej technológii}
\item \textbf{PB2} - Metodika, plán výskumu\\
Výber zariadení na experimenty, návrh výskumného postupu, definícia testovacích parametrov.

\textbf{Milník č.2: Pripravená metodika merania parametrov štandardu (výkon, dosah, energetická efektivita)}\\
\end{itemize}
Výstupom bude prehľadový článok, ktorý môže byť publikovateľný v odborných článkoch.
}
\item {\textbf{2. Fáza - Experimentálne testovanie a analýza:}\\
\begin{itemize}
\item \textbf{PB3} - Implementácia, zostrojenie testovacieho prostredia\\
\textbf{Milník č.3: Funkčné prostredie pre testovanie}\\
\item \textbf{PB4} - Experimentálne testovanie a porovnanie LPWAN technológií\\
Meranie dosahu, latencie, rýchlosti, energetickej náročnosti a stability signálu.\\
\textbf{Milník č.4: Dokončené testovanie a získané dáta}\\
\item \textbf{PB5} - Komparatívna analýza\\
Porovanie HaLow so Sigfox,Lora z merateľných hladísk.
\textbf{Milník č.5: Vyhotovená správa analýzy}\\
\end{itemize}
}
Výstupom bude technická správa, prezentácie pre APVV, štatistické tabuľky pre interné použitie a návrh vedeckého článku
\item {\textbf{3. Fáza - Finalizácia projektu, aplikácia a publikácia:}\\
\begin{itemize}
\item \textbf{PB6} - Overenie výsledkov\\
Overenie konzistentnosti dát.\\
\textbf{Milník č.6: Overená presnosť výsledkov}\\
\item \textbf{PB7} - Návrhy aplikácií a implementácií Wi-fi HaLow\\
Definovanie oblastí, kde by sa dalo HaLow aplikovať.\\
\textbf{Milník č.7: Návrhy pre slovenskú infraštruktúru}\\
\item \textbf{PB8} - Publikácia výsledkov\\
Publikovanie vedeckých článkov, prezentácie na prednáškach, konferenciách a v odborných médiách.\\
\textbf{Milník č.8: Publikované výstupy, mediálne výstupy}\\
\end{itemize}
Výstupom bude finalná správa projektu, vedecké publikácie, odporúčania pre budúci výskum a implementáciu IoT na Slovensku\\
}
\end{itemize}
}
\item {Riadenie projektu bude rozdelené do štyroch následujúcich úrovní:}

\begin{itemize}
\item Hlavný zodpovedný riešiteľ projektu - Strategické vedenie, komunikácia s APVV, schvaľovanie výstupov, rozpočtové plánovanie
\item Zástupca hlavného riešiteľa - koordinuje výskumný tím, vedie interné stretnutia, dozerá na napĺňanie milníkov
\item Výskumný tím (4 hlavní a 27 zamestnancov) - sú tu zahrnutí aj odborníci na jednotlivé časti projektu - sieťové technológie, dátová analytika, štatistika, IoT odborníci.
\item Administratíva a technická podpora - účtovníctvo, materiál, technické vybavenie, záznamy o projekte
\end{itemize}

Projekt bude riadený nasledovnými procesmi:
\begin {itemize}
\item Pravidelné mesačné stretnutia - vyhodnocovanie pribehu projektu, plánovanie nasledujucého mesiaca
\item Štvrťročné správy - sumár výsledkov, finančných tokov
\item Interný informačný systém - Git: zdielanie dokumentov, výsledkov a interných správ
\item Spolupráca s APVV a externá kontrola - dvakrát ročne sa uskutoční oficiálne stretnutie s APVV, kde sa zhodnotí pokrok.
\end {itemize}

\item Riziká implementácie projektu a spôsoby zmierenia:\\
 Riziká môžu vzniknúť v technickej, organizačnej aj vo finančnej oblasti projektu:
\begin{itemize}
\item Technické riziko - Nedostupnosť komponentov alebo zariadení HaLow\\
\textbf{Vytvorenie simulácií vnútri virtuálneho prostredia, alternatívne testovacie prostredia}

\item Organizačné riziko - Nedostatočná komunikácia, neskoré dodanie dát, nestíhanie termínov\\
\textbf{Zavedenie interného manažmentu úloh - Jira alebo pravidelné reporty výskumu}

\item Finančné riziko - nedostatok financií, prekročenie rozpočtu\\
\textbf{Rezervný fond, revízia výdavkov}

\item Publikačné riziko - Odmietnutie vedeckého článku recenzentmi\\
\textbf{Príprava alternatívnych publikácií a zapojenie sa do rôznych technických konferencií}
\end {itemize}

\item Adekvátnosť navrhnutého rozpočtu\\

Navrhnutý rozpočet je vo výške 300 000 €, je primeraný dĺžke trvania (3 roky), rozsahu a aj technickej náročnosti cielov
Finančná náročnosť je najmä v nákupe testovacej výbavy a v personálnej podpore počas celých troch rokov

\begin{itemize}
\item Mzdové a osobné náklady -- 110 tisíc € a odvody -- 36 tisíc €\\
Čiastočné úväzky výskumného tímu sú dostatočné, aby boli pokryté osobnými nákladmi a sociálnych, či zdravotných odvodov. - priblížne 50 tisíc €\\
Bude potrebné, aby bolo aj 30 zamestnancov adekvátne odmenených za meranie/testovanie/programovanie... - priblížne 60 tisíc €\\

\item Materiál -- 53 tisíc €\\
Nákup HaLow modulov, antén, meracích pristrojov (spektrálne analyzátory, generátory signálu, power metery, komunikačné analyzátory)\\
Umožňuje overenie výkonu, dosahu a efektívnosti v približne reálnych podmienkach IoT komunikácie.

\item Služby -- 20 tisíc €\\
Externá analýza - (Excel/ChatGPT API), Kalibrácie prístrojov, prenájmy špeciálnych priestrojov na testovanie\\
Niektoré z meraní potrebujú platené externé služby alebo špecializované softvéry (MATLAB)\\

\item Cestovné náklady -- 17 tisíc €\\
Konferencie, testovania v iných prostrediach (mestá, polia a pod.)\\

\item Prevádzkové náklady + nepriame -- približne 40 tisíc €\\
Prevádzka serverových zariadení, sieťové poplatky, údržba, administratívna podpora, účtovnícke služby\\

\item Popularizácia a publikácie -- približne 13 tisíc €\\
Príprava publikácií, ttvorba prezentačných materiálov a videí\\
Poplatky za články konferencie, stánky sa pohybujú od tisícov eur, \\


\end {itemize}

\item Infraštruktúra pracoviska - STU FIIT\\
Fakulta informatiky a informačných technológií Slovenskej Technickej Univerzity ponúka rozvinutú výskumnú infraštruktúru v oblastí, ktorých sa náš výskum dotýka.\\ Menovito sieťové technológie. Fakulta je vybavená najmodernejšími zariadeniami, vhodnými na skúmanie HaLow technológie a jej testovanie v areáloch laboratória sieťových technológií.\\
Fakulta disponuje odborníkmi v oblasti nášho výskumu: bezdrôtové systémy.\\
STU FIIT je teda vhodným priestorom na skúmanie problematiky Wi-fi HaLow.

\end{itemize}


\end{document}
